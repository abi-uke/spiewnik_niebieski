%w Znów od świtu kręci się ziemia
%r Płyną obłoki w niebieski rejs
\documentclass[a5paper]{article}
 \usepackage[english,bulgarian,russian,ukrainian,polish]{babel}
 \usepackage[utf8]{inputenc}
 %\usepackage{polski}
 \usepackage[T1]{fontenc}
 \usepackage[margin=1.5cm]{geometry}
 \usepackage{multicol}
 \setlength\columnsep{10pt}
 \begin{document}
 %\pagenumbering{gobble}


\noindent
\fontsize{12pt}{15pt}\selectfont
\textbf{Niebieski rejs} \\
\fontsize{8pt}{10pt}\selectfont
Browar Żywiec / s. i muz. Jerzy Reiser \\ \\
\fontsize{10pt}{12pt}\selectfont
\leftskip0cm
\begin{tabular}{@{}p{8.00cm}p{3cm}@{}}
\noindent
Znów od świtu kręci się ziemia & e \\
Noc ciągle goni za dniem & C e \\
A godziny przed nami wciąż idą bez końca & D G  D  C G  D \\ \\
 
Wrzesień lato na jesień zamienia \\
Jak miłość pocztówkę pod szkłem \\
Tylko czasem coś w górę nas niesie do słońca \\ \\
\end{tabular}

\leftskip1cm
\noindent
\begin{tabular}{@{}p{7.00cm}p{3cm}@{}}
C G  D e  a C  D \\ \\
Płyną obłoki w niebieski rejs & C G  D e \\
W błękit głęboki, w przestrzeni sens & C G  D e \\
Skąd je sprowadza odwieczny wiatr & C G  D e \\
I gdzie wędrują, gdzie pędzą tak & a C  D \\ \\

I chociaż dłużej nam trzeba żyć & C G  D e \\
Flotą obłoków musimy być & C G  D e \\
Jaki też po nas zostanie ślad & C G  D e \\
Kiedy zawieje odwieczny wiatr & a G C  D G \\ \\
\end{tabular}

\leftskip0cm
\noindent
\begin{tabular}{@{}p{7.00cm}p{3cm}@{}}
Coraz cieńsze słoje są w drzewach \\
I przyjdą znów drzewa nam ściąć \\
Nie czekając aż liście zabierze listopad \\ \\
 
Coraz rzadziej nas słowo rozgrzewa \\
I ciężko ze sobą go wziąć \\
Nie tak łatwo już dzisiaj pobujać w obłokach \\ \\
\end{tabular}

\leftskip1cm
\noindent
\begin{tabular}{@{}p{7.00cm}p{3cm}@{}} 
Płyną obłoki… | x2\\
\end{tabular}

\end{document}
