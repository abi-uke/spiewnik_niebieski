%w Znowu czas odejść w góry
%r A my się dalej pniemy w jesienie
\documentclass[a5paper]{article}
 \usepackage[english,bulgarian,russian,ukrainian,polish]{babel}
 \usepackage[utf8]{inputenc}
 %\usepackage{polski}
 \usepackage[T1]{fontenc}
 \usepackage[margin=1.5cm]{geometry}
 \usepackage{multicol}
 \setlength\columnsep{10pt}
 \begin{document}
 %\pagenumbering{gobble}


\noindent
\fontsize{12pt}{15pt}\selectfont
\textbf{Widok z Lackowej} \\
\fontsize{8pt}{10pt}\selectfont
A. Wierzbicki \\ \\
\fontsize{10pt}{12pt}\selectfont
\leftskip0cm
\begin{tabular}{@{}p{8.50cm}p{3cm}@{}}
\noindent
Znowu czas odejść w góry & a H7 e \\
i po co nam była ta miłość, & C D G \\
widziana w twarzach panien płowych. & a H7 e \\ \\
Znowu czas odejść w góry & a H7 e \\
i prędko zapomnieć jak pachnie wino & C D G \\
z czerwonych głogów, z jałowca herbem sinym. & Fis H7 \\ \\
\end{tabular}

\leftskip1cm
\noindent
\begin{tabular}{@{}p{7.50cm}p{3cm}@{}}
A my się dalej pniemy w jesienie, & C D G e \\
wsłuchani w echa marzeń nie dorosłych, & C D G \\
potem się damy oszukać od nowa & a D G e \\
nadchodzącym wiosnom. & Fis H7 \\ \\
\end{tabular}

\leftskip0cm
\noindent
\begin{tabular}{@{}p{7.50cm}p{3cm}@{}}
Znowu czas odejść z gór \\
i po co nam była ta miłość, \\
do białych wiosek i ciszy przy drodze. \\
Znowu czas odejść z gór \\ \\
i prędko zapomnieć jak pachnie dymem \\
dzień obudzony \\
w cerkiewnych bani łkaniu. \\ \\
\end{tabular}

\leftskip1cm
\noindent
\begin{tabular}{@{}p{7.50cm}p{3cm}@{}}
A my się dalej pniemy w jesienie…
\end{tabular}

\end{document}
