%r A ziemia toczy, toczy swój garb uroczy
\documentclass[a5paper]{article}
 \usepackage[english,bulgarian,russian,ukrainian,polish]{babel}
 \usepackage[utf8]{inputenc}
 %\usepackage{polski}
 \usepackage[T1]{fontenc}
 \usepackage[margin=1.5cm]{geometry}
 \usepackage{multicol}
 \setlength\columnsep{10pt}
 \begin{document}
 %\pagenumbering{gobble}


\noindent
\fontsize{12pt}{15pt}\selectfont
\textbf{Opadły mgły, wstaje nowy dzień} \\
\fontsize{8pt}{10pt}\selectfont
Stare Dobre Małżeństwo / sł. Edward Stachura, muz. Krzysztof Myszkowski \\ \\
\fontsize{10pt}{12pt}\selectfont
\leftskip0cm
\begin{tabular}{@{}p{8.50cm}p{3cm}@{}}
\noindent
Opadły mgły i miasto ze snu się budzi, & D G \\
Górą czmycha już noc, & D A \\
Ktoś tam cicho czeka by ktoś powrócił; & D G \\
Do gwiazd jest bliżej niż krok! & D A\\
Pies się włóczy popod murami bezdomny; & D G \\
Niesie się tęsknota czyjaś & D A \\
Na świat cztery strony! & D \\ \\
\end{tabular} 

\leftskip1cm
\noindent
\begin{tabular}{@{}p{7.50cm}p{3cm}@{}}
A Ziemia toczy, toczy swój garb uroczy; \\
Toczy, toczy się los! \\
Ty co płaczesz, ażeby śmiać mógł się ktoś \\
- Już dość! Już dość! Już dość! \\
Odpędź czarne myśli! Dość już twoich łez! \\
Niech to wszystko przepadnie we mgle! \\
Bo nowy dzień wstaje, \\
Bo nowy dzień wstaje, \\
Nowy dzień! \\ \\
\end{tabular} 

\leftskip0cm
\noindent
\begin{tabular}{@{}p{7.50cm}p{3cm}@{}}
Z dusznego snu już miasto się wynurza, \\
Słońce wschodzi gdzieś tam, \\
Tramwaj na przystanku zakwitł jak róża; \\
Uchodzą cienie do bram! \\
Ciągną swoje wózki-dwukółki mleczarze;  \\
Nad dachami snują się sny podlotków pełne marzeń! \\ \\
\end{tabular} 

\leftskip1cm
\noindent
\begin{tabular}{@{}p{7.50cm}p{3cm}@{}}
A Ziemia toczy, toczy swój garb uroczy; \\
Toczy, toczy się los! \\
Ty co płaczesz, ażeby śmiać mógł się ktoś \\
- Już dość! Już dość! Już dość! \\
Odpędź czarne myśli! Porzuć błędny wzrok! \\
Niech to wszystko zabierze już noc! \\
Bo nowy dzień wstaje, \\
Bo nowy dzień wstaje, \\
Nowy dzień! \\ \\
\end{tabular} 

\end{document}
