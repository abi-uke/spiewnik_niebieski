%w Polami, polami, po miedzach, po miedzach
\documentclass[a5paper]{article}
 \usepackage[english,bulgarian,russian,ukrainian,polish]{babel}
 \usepackage[utf8]{inputenc}
 %\usepackage{polski}
 \usepackage[T1]{fontenc}
 \usepackage[margin=1.5cm]{geometry}
 \usepackage{multicol}
 \setlength\columnsep{10pt}
 \begin{document}
 %\pagenumbering{gobble}


\noindent
\fontsize{12pt}{15pt}\selectfont
\textbf{Nuta z Ponidzia} \\
\fontsize{8pt}{10pt}\selectfont
Wolna Grupa Bukowina / sł. i muz.: Wojciech Bellon \\ \\
\fontsize{10pt}{12pt}\selectfont
\leftskip0cm
\begin{tabular}{@{}p{7.50cm}p{3cm}@{}}
\noindent
a F7+ G C7+ \\
d7 G C \\
h7 E7/4 \\
a G6 F7+ G6 \\
a G e E \\ \\
Polami, polami, po miedzach, po miedzach & a F G C7+ \\
Po błocku skisłym w mgłę i wiatr & d7 G C7+ \\
Nie za szybko, kroki drobiąc & h7 E7 \\
Idzie wiosna, idzie nam & a G6 F7+ G \\
Idzie wiosna, idzie... & a G e E \\
...nam & a F E \\ \\
\end{tabular}

\leftskip0cm
\noindent
\begin{tabular}{@{}p{7.50cm}@{}}
Łąkami, łąkami, po lasach, po lasach \\
To zapatrzona w słońca blask \\
To się w wodzie przeglądając \\
Idzie wiosna, idzie nam. \\ \\
\end{tabular}

\leftskip0cm
\noindent
\begin{tabular}{@{}p{7.50cm}@{}}
Rozłożyła wiosna spódnicę zieloną \\
Przykryła błota bury błam \\
Pachnie ziemia ciałem młodym \\
Póki wiosna, póki trwa \\ \\
\end{tabular}

\leftskip0cm
\noindent
\begin{tabular}{@{}p{7.50cm}@{}}
Rozpuściła wiosna warkocze kwieciste \\
Zbarwniały łąki niczym kram \\
Będzie odpust pod Wiślicą \\
Póki wiosna, póki trwa \\ \\
\end{tabular}

\leftskip0cm
\noindent
\begin{tabular}{@{}p{7.50cm}p{3cm}@{}}
Ponidzie wiosenne, Ponidzie leniwe & a F G C \\
Prężysz się jak do słońca kot & d7 G C\\
rozciągnięte na tych polach & h7 E7 \\
Lichych lasach, pstrych łozinach & h7 E7 \\
Skałkach słońcem rozognionych & h7 E7 \\
Nidą w łąkach roziskrzoną & h7 E7 \\
Na Ponidziu wiosna trwa & a G F E \\ 
Na Ponidziu wiosna trwa & a G F E \\ 
Na Ponidziu... & a G F E \\
... wiosna trwa! & a \\
\end{tabular}

\end{document}
