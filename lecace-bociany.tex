%w Obudzić się rosą rozkażę
%r Wtedy ręce rozłożę jak bociek
\documentclass[a5paper]{article}
 \usepackage[english,bulgarian,russian,ukrainian,polish]{babel}
 \usepackage[utf8]{inputenc}
 %\usepackage{polski}
 \usepackage[T1]{fontenc}
 \usepackage[margin=1.5cm]{geometry}
 \usepackage{multicol}
 \setlength\columnsep{10pt}
 \begin{document}
 %\pagenumbering{gobble}


\noindent
\fontsize{12pt}{15pt}\selectfont
\textbf{Lecące bociany} \\
\fontsize{8pt}{10pt}\selectfont
U Pana Boga za Piecem / Sylwek Szweda \\ \\
\fontsize{10pt}{12pt}\selectfont
\leftskip0cm
\begin{tabular}{@{}p{10.50cm}p{3cm}@{}}
\noindent
( G-D-G G-D-e C h7 a7 D7 ) ( G-D-G G-D-e C h7 a7 D7 ) x2 \\ \\
\end{tabular}  

\leftskip0cm
\noindent
\begin{tabular}{@{}p{7.50cm}p{3cm}@{}}   
Obudzić się rosie rozkażę, & G D G G D e \\
Nawet gdy dzień zaśpi, & C h7 a7 D7 \\
A kiedy już wstaną pejzaże & G D G G D e \\
I zakwitnie jaśmin: & C h7 a7 D7 \\ \\
\end{tabular}

\leftskip1cm
\noindent
\begin{tabular}{@{}p{6.50cm}p{3cm}@{}}
Wtedy ręce rozłożę jak bociek & C D e G D \\
I jak Chrystus & C D e \\
Zastygnę w locie, & G D \\
Spojrzę na góry & C D e \\
Jak na piersi dziewczęce & G D \\
I znów jak bociek rozłożę ręce. & C D7 \\ \\
\end{tabular}

\leftskip0cm
\noindent
\begin{tabular}{@{}p{8.50cm}@{}}
Słońce przywitam jak gospodarz domu, \\
W którym garnki nie płaczą. \\
Zasieję pieśni i nie zdradzę nikomu \\
Ile dla mnie znaczą. \\ \\
\end{tabular}

\leftskip1cm
\noindent
\begin{tabular}{@{}p{7.50cm}@{}}
Tylko ręce rozłożę jak bociek…
\end{tabular}

\leftskip0cm
\noindent
\begin{tabular}{@{}p{8.50cm}@{}}
A kiedy noc uroczyście oblecze \\
Swój czarny garnitur, \\
Rozpalę ogień i zaproszę wędrowców - \\
Pośpiewamy do świtu. \\ \\
\end{tabular}

\leftskip1cm
\noindent
\begin{tabular}{@{}p{7.50cm}@{}}
Tylko ręce rozłożę jak bociek… \\ \\
\end{tabular}

\leftskip0cm
\noindent
\begin{tabular}{@{}p{8.50cm}@{}}
solo: ( C h7 / e - C h7 / C h7 / C D7 ) x2 \\ \\
\end{tabular}

\leftskip1cm
\noindent
\begin{tabular}{@{}p{7.50cm}@{}}
Tylko ręce rozłożę jak bociek… \\ \\
\end{tabular}
 
\leftskip0cm
\noindent
\begin{tabular}{@{}p{8.50cm}@{}}
( G-D-G G-D-e C h7 a7 D7 ) x2 G D G
\end{tabular}

\end{document}
