%w Orzech starzeje się po włosku
\documentclass[a5paper]{article}
 \usepackage[english,bulgarian,russian,ukrainian,polish]{babel}
 \usepackage[utf8]{inputenc}
 %\usepackage{polski}
 \usepackage[T1]{fontenc}
 \usepackage[margin=1.5cm]{geometry}
 \usepackage{multicol}
 \setlength\columnsep{10pt}
 \begin{document}
 %\pagenumbering{gobble}


\noindent
\fontsize{12pt}{15pt}\selectfont
\textbf{Z podwórka} \\
\fontsize{8pt}{10pt}\selectfont
Stare Dobre Małżeństwo \\ \\
\fontsize{10pt}{12pt}\selectfont
\leftskip0cm
\begin{tabular}{@{}p{6.50cm}p{3cm}@{}}
\noindent
Przech starzeje się po włosku & D fis G A \\
Jest twardy - lecz jednak do zgryzienia & h h7 G A \\
Zielone krople jego oczu & h A G fis \\
Już mało mają do patrzenia & G A D \\ \\

Sąsiadka mruga do mnie często \\
To tik co został jej z małżeństwa \\
A mąż pijany suszy się na sznurze \\
Już wkrótce przyjdą znowu święta \\\\

Wisi też dywan co latać zapomniał \\
Prezent od teściów - wierny pies \\
Sierść mu się jeży już niegęsta \\
Rzadkim wzruszeniom szkoda łez \\\\

Tu trzeba twardo - jak ten orzech \\
Niech nie pomyślą nic sąsiedzi \\
Z dywanu lecą chmury kurzu \\
To może w nim złe licho siedzi?
\end{tabular}

\end{document}
