%w Słyszałem jego krok wdłuż mojej kamienicy
%r Mój darmodziej, miłość gra, wierny naszym losom
\documentclass[a5paper]{article}
 \usepackage[english,bulgarian,russian,ukrainian,polish]{babel}
 \usepackage[utf8]{inputenc}
 %\usepackage{polski}
 \usepackage[T1]{fontenc}
 \usepackage[margin=1.5cm]{geometry}
 \usepackage{multicol}
 \setlength\columnsep{10pt}
 \begin{document}
 %\pagenumbering{gobble}


\noindent
\fontsize{12pt}{15pt}\selectfont
\textbf{Darmodziej} \\
\fontsize{8pt}{10pt}\selectfont
sł. i muz. Jaromir Nohavica, tłum. Antoni Muracki \\ \\
\fontsize{10pt}{12pt}\selectfont
\leftskip0cm
\begin{tabular}{@{}p{9cm}p{3cm}@{}}
\noindent
Słyszałem jego krok wzdłuż mojej kamienicy & a e a e \\
Utkwiłem chciwy wzrok, gdy kroczył po ulicy & a e a e \\
Chorałem flet mu brzmiał donośnie - niczym dzwon & C G a \\
I był w tym cały żal i czerń tysiąca wron & e F \\
I wtedy zrozumiałem nagle - że to on, że to on & F0 E a \\ \\

Na boso zbiegłem w dół wyszedłem mu naprzeciw \\
W podwórzu głodny szczur buszował w stercie śmieci \\
Przy bokach ciepłych żon gdzie chuć z miłością śpi \\
wtuleni w kołdry schron rodzinny grali film \\
A ja tak chciałem znać odpowiedź - dokąd iść, dokąd iść... \\ \\
\end{tabular}

\leftskip1cm
\noindent
\begin{tabular}{@{}p{6cm}p{3cm}@{}}
Na na na… & a e C G a F F0 E a \\ \\
\end{tabular}

\leftskip0cm
\noindent
\begin{tabular}{@{}p{9.5cm}@{}}
Po bruku gnałem w dół, żeby mu zabiec drogę \\ 
Miał płaszcz z wężowych skór i wokół wiało chłodem \\
Obrócił ku mnie twarz, swe oczy pełne wron \\
A blizny dawny blask skrywały niby szron \\
I wtedy zrozumiałem nagle, kim był on, kim był on \\ \\

Ze strachu ledwo szedł i słaniał się jak kloszard \\
Do ust przykładał flet od Hieronima Boscha \\
W niebie się księżyc tlił jak lampa w ciemną noc \\
jak mój sumienia krzyk, gdy rzyga schlane w sztok  \\
I że to jest Darmodziej, czułem że to właśnie on, właśnie on \\ \\
\end{tabular}

\leftskip1cm
\noindent
\begin{tabular}{@{}p{8.5cm}p{3cm}@{}}
Mój Darmodziej, miłość gra, wierny naszym losom \\
Bard, który zna wszystkie sny i przemyka nocą \\
Mój Darmodziej, słodki grzech z jadem pod językiem, \\
Gdy sprzedać chce to co ma - igły ze słownikiem. \\ \\
\end{tabular}

\leftskip0cm
\noindent
\begin{tabular}{@{}p{8.5cm}p{3cm}@{}}
Przez miasto wczoraj szedł, ot - zwykły domokrążca, \\
A drogi jego kres znaczyła krwawa wstążka \\
Flet jego wziąłem ja, a brzmiał mi niczym dzwon \\
i był w nim cały żal i czerń tysiąca wron \\
I wtedy już wiedziałem,że ja to on, ja to on \\ \\
\end{tabular}

\leftskip1cm
\noindent
\begin{tabular}{@{}p{8.5cm}p{3cm}@{}}
Wasz Darmodziej, miłość gram, wiernym waszym losom \\
Bard który zna wszystkie sny i przemyka nocą \\
Wasz Darmodziej, słodki grzech, z jadem pod językiem \\
Gdy sprzedać chcę to co mam - igły ze słownikiem.
\end{tabular}

\end{document}
