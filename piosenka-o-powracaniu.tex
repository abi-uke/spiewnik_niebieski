%w Powrócę kiedyś w miast ciemny gotyk
%r Znów będę kłaniał się starym gazowym latarniom
\documentclass[a5paper]{article}
 \usepackage[english,bulgarian,russian,ukrainian,polish]{babel}
 \usepackage[utf8]{inputenc}
 %\usepackage{polski}
 \usepackage[T1]{fontenc}
 \usepackage[margin=1.5cm]{geometry}
 \usepackage{multicol}
 \setlength\columnsep{10pt}
 \begin{document}
 %\pagenumbering{gobble}


\noindent
\fontsize{12pt}{15pt}\selectfont
\textbf{Piosenka o powracaniu (Latarnie, Gazowe latarnie)} \\
\fontsize{8pt}{10pt}\selectfont
autor nieznany \\ \\
\fontsize{10pt}{12pt}\selectfont
\leftskip0cm
\begin{tabular}{@{}p{9.00cm}p{3cm}@{}}
\noindent
Powrócę kiedyś w miast ciemny gotyk & C C7 \\
Z gór szerokich jasnych łąk & F G \\
Aby zrozumieć mowę tęsknoty & C7 \\
Aby odetchnąć ciepłem twoim rąk & F G C \\
Zabiorę z sobą zapach słońca & E a \\
I blask poranka roześmiany & F G C \\
I koncert naszych strun bez końca & E a \\
Na pieciolinii zapisanych & F G \\ \\
\end{tabular}

\leftskip1cm
\noindent
\begin{tabular}{@{}p{8.00cm}p{3cm}@{}}
Znów będę kłaniał się starym gazowym latarniom & C F G C \\
Zarzucę drzewom odpowiedź na szum powitania & F G C \\
Zajrzę do okien witrynom i nocnym kawiarniom & F G C \\
A może zebrę nakarmię na skrzyżowaniu & F G C \\ \\
\end{tabular}

\leftskip0cm
\noindent
\begin{tabular}{@{}p{9.00cm}@{}}
Pokażę ci jak pachną latarnie \\
Jak się noc zanurza w rzeki czerń \\
A kiedy jasna mgła nas ogarnie \\
Pierwszy tramwaj nam przyniesie dzień \\
Przez barodkowy park pójdziemy \\
Zanim do domu powrócimy \\
Liczyć będziemy dni jesieni \\
Aby do zimy aby do zimy \\ \\
\end{tabular}

\leftskip1cm
\noindent
\begin{tabular}{@{}p{8.00cm}@{}}
Znów będę…
\end{tabular}

\end{document}
