%w Polsko - ruska Madonna
\documentclass[a5paper]{article}
 \usepackage[english,bulgarian,russian,ukrainian,polish]{babel}
 \usepackage[utf8]{inputenc}
 %\usepackage{polski}
 \usepackage[T1]{fontenc}
 \usepackage[margin=1.5cm]{geometry}
 \usepackage{multicol}
 \setlength\columnsep{10pt}
 \begin{document}
 %\pagenumbering{gobble}


\noindent
\fontsize{12pt}{15pt}\selectfont
\textbf{Madonna} \\
\fontsize{8pt}{10pt}\selectfont
Wątli Kołodzieje \\ \\
\fontsize{10pt}{12pt}\selectfont
\leftskip0cm
\begin{tabular}{@{}p{8.00cm}p{3cm}@{}}
\noindent
Polsko – ruska Madonna. & F G a \\
Polsko – starocerkiewna, & F G a \\
z gorejącego złota, & F G a \\
z żywego drewna, srebra. & F G a \\ \\
\end{tabular}

\leftskip0cm
\noindent
\begin{tabular}{@{}p{8.00cm}p{3cm}@{}}
Bizantyjsko – słowiańska, & F G d \\
w śnieżnobiałych sukienkach. & F G d \\
Nie na bielonych płótnach, & F G d \\
na trumiennych, na trumiennych deskach. & C G \\ \\
\end{tabular}

\leftskip0cm
\noindent
\begin{tabular}{@{}p{8.00cm}@{}}
Nie bojarska, nie carska, \\
zasmucona kamienna twarz. \\
Nie bojarska, nie carska. \\
Zasmucona, chłopska ikona.
\end{tabular}

\end{document}
