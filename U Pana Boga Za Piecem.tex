%w U Pana Boga Za Piecem
%r I „Kipiatok, wrzątek kipi Ameryka!”
\documentclass[a5paper]{article}
 \usepackage[english,bulgarian,russian,ukrainian,polish]{babel}
 \usepackage[utf8]{inputenc}
 %\usepackage{polski}
 \usepackage[T1]{fontenc}
 \usepackage[margin=1.5cm]{geometry}
 \usepackage{multicol}
 \setlength\columnsep{10pt}
 \begin{document}
 %\pagenumbering{gobble}


\noindent
\fontsize{12pt}{15pt}\selectfont
\textbf{U Pana Boga Za Piecem} \\
\fontsize{8pt}{10pt}\selectfont
Sodki Caus od Buby, sł., muz.: K. Jurkiewicz \\ \\
\fontsize{10pt}{12pt}\selectfont
\leftskip0cm
\begin{tabular}{@{}p{9.5cm}p{3cm}@{}}
\noindent
U Pana Boga za piecem, jak to za piecem, & G C \\
Czas się powoli telepie, & G D \\
Błotnistą drogą na wozie, deszczowa jesień & G C \\
Chmurę za chmurą wiezie, & G D \\
Okna za szarą zasłoną, a góry w deszczu & e C \\
Wciąż nie przestają kusić, & D \\
Że jutro będzie pogoda, jutro na pewno, & G C \\
Jutro już można wyruszyć... & G D \\ \\
\end{tabular}

\leftskip1cm
\noindent
\begin{tabular}{@{}p{8.5cm}p{3cm}@{}}
I „Kipiatok, wrzątek kipi Ameryka!” & C D G \\
Kolejny dzień odmierzany litrami herbaty powoli znika. & C D e/G \\ \\
\end{tabular}

\leftskip0cm
\noindent
\begin{tabular}{@{}p{8.5cm}p{3cm}@{}}
U Pana Boga za piecem lekko fałszywa \\
Gitara kaleczy uszy, \\
Na elektrycznych grzejnikach zajęły miejsca \\
Swetry, skarpety i buty. \\
Ludzie z mgły wolno wracają ze swojej bazy \\
Z chmielową nadzieją w duszy \\
że jutro będzie pogoda, jutro na pewno \\
Jutro już można wyruszyć... \\ \\

U Pana Boga za piecem, robi się pusto \\
Na każde słońca skinienie \\
Dopiero wieczór zaprasza zmęczone kroki, \\
Już słychać jak dudnią w sieni... \\
I każdy skrawek podłogi zmienia się teraz \\
W ogromne snów kartoflisko, \\
A góry rosną tuż obok, a gwiazd gromady \\
Podchodzą naprawdę blisko... 
\end{tabular}

\end{document}
