%w Żegnaj nam dostojny stary porcie
%r A więc - żegnaj mi, kochana ma
\documentclass[a5paper]{article}
 \usepackage[english,bulgarian,russian,ukrainian,polish]{babel}
 \usepackage[utf8]{inputenc}
 %\usepackage{polski}
 \usepackage[T1]{fontenc}
 \usepackage[margin=1.5cm]{geometry}
 \usepackage{multicol}
 \setlength\columnsep{10pt}
 \begin{document}
 %\pagenumbering{gobble}


\noindent
\fontsize{12pt}{15pt}\selectfont
\textbf{Pożegnanie Liverpoolu} \\
\fontsize{8pt}{10pt}\selectfont
sł. Krzysztof Kuza, Jerzy Rogacki, muz. tradycyjna \\ \\
\fontsize{10pt}{12pt}\selectfont
\leftskip0cm
\begin{tabular}{@{}p{7.50cm}p{3cm}@{}}
\noindent
Żegnaj nam dostojny stary porcie & C F C (D D7 G D) \\
Rzeko Mersey żegnaj nam & G (A A7) \\
Wypływamy dziś na rejs do Californii & C F C (D D7 G D) \\
Byłem tam już niejeden raz & G C (A D) \\ \\
\end{tabular}

\leftskip1cm
\noindent
\begin{tabular}{@{}p{6.50cm}p{3cm}@{}}
A więc - żegnaj mi, kochana ma & C G F C (A G D) \\
Już za chwilę wypłyniemy w długi rejs & G (A A7) \\
Ile miesięcy Cię nie będę widział & C C7 (D D7) \\
Nie wiem jam & F C (G D) \\
Lecz pamiętać zawsze będę Cię & G C (A7 D) \\ \\
\end{tabular}

\leftskip0cm
\noindent
\begin{tabular}{@{}p{8.50cm}@{}}
Zaciągnąłem się na herbaciany kliper \\
DObry statek choć sławę ma złą \\
A, że kapitanem jest tam stary Burges \\
Pływającym piękłem wszyscy go zwą \\ \\
\end{tabular}

\leftskip1cm
\noindent
\begin{tabular}{@{}p{7.50cm}@{}}
A więc… \\ \\
\end{tabular}

\leftskip0cm
\noindent
\begin{tabular}{@{}p{8.50cm}@{}}
Z kapitanem tym płynę już nie pierwszy raz \\
Znamy się od wielu, wielu lat \\
Jeśliś dobrym żeglarzem radę sobie dasz \\
Jeśli nie toś cholernie wpadł \\ \\
\end{tabular}

\leftskip1cm
\noindent
\begin{tabular}{@{}p{7.50cm}@{}}
A więc… \\ \\
\end{tabular}

\leftskip0cm
\noindent
\begin{tabular}{@{}p{8.50cm}p{3cm}@{}}
Zegnaj nam dostojny stary porcie \\
Rzeko Mersey żegnaj nam \\
Wypływamy już na rejs do Californi \\
Gdy wrócimy opowiemy Wam \\ \\
\end{tabular}

\leftskip1cm
\noindent
\begin{tabular}{@{}p{7.50cm}@{}}
	A więc… \\ \\
\end{tabular}
\end{document}
