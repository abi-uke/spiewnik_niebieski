%w Siedzisz na swym pniaczku jak bieszczadzki gazda
%r Wskaż nam Panie drogę po Bieszczadzkich szlakach
\documentclass[a5paper]{article}
 \usepackage[english,bulgarian,russian,ukrainian,polish]{babel}
 \usepackage[utf8]{inputenc}
 %\usepackage{polski}
 \usepackage[T1]{fontenc}
 \usepackage[margin=1.5cm]{geometry}
 \usepackage{multicol}
 \setlength\columnsep{10pt}
 \begin{document}
 %\pagenumbering{gobble}


\noindent
\fontsize{12pt}{15pt}\selectfont
\textbf{Chrystus Bieszczadzki} \\
\fontsize{8pt}{10pt}\selectfont
Cisza Jak Ta / sł. K. Patora, muz. M. Łangowski \\ \\
\fontsize{10pt}{12pt}\selectfont
\leftskip0cm
\begin{tabular}{@{}p{9.00cm}p{3cm}@{}}
\noindent
Siedzisz na swym pniaczku jak bieszczadzki gazda & h A \\
Błogosławisz ptakom wracającym do gniazda & G D A \\
Tym, co przyszli do Ciebie bo z serca chcieli & h A \\
I tym co wśród pożogi odejść stąd musieli & G A \\ \\
\end{tabular}

\leftskip1cm
\noindent
\begin{tabular}{@{}p{8.00cm}p{3cm}@{}}
Wskaż nam Panie drogę po Bieszczadzkich szlakach & D - D/h/A \\
Zagubionym – bądź echem w strumieniach i ptakach & G - G/A/D \\
I światłem w ciemności, jak twój księżyc blady & D - D/h/A \\
Gdzie umilkły cerkwie i zdziczały sady & G - G/A/D \\ \\
\end{tabular}

\leftskip0cm
\noindent
\begin{tabular}{@{}p{9.00cm}p{3cm}@{}}
Tym, co przyszli tutaj by prawem zwyczaju & h A \\
Podziękować  Tobie za przedsionek raju & G D A \\
Za ptasie koncerty o porannym brzasku & h A \\
I za lipcowe noce przy księżyca blasku & G A \\ \\
\end{tabular}

\leftskip1cm
\noindent
\begin{tabular}{@{}p{8.00cm}p{3cm}@{}}
Wskaż nam Panie drogę po Bieszczadzkich szlakach & D - D/h/A \\
Zagubionym – bądź echem w strumieniach i ptakach & G - G/A/D \\
I światłem w ciemności, jak twój księżyc blady & D - D/h/A \\
Gdzie umilkły cerkwie i zdziczały sady & G - G/A/D \\ \\
\end{tabular}

\leftskip0cm
\noindent
\begin{tabular}{@{}p{9.00cm}p{3cm}@{}}
Zieleń skryła blizny – zostały wspomnienia & cis H \\
W sercach  został smak tamtego cierpienia & A E H \\
Znad tych samych ognisk inne pieśni płyną & cis H \\
Gnane ciepłym wiatrem do wzgórz nad Soliną & A H \\ \\
\end{tabular}

\leftskip1cm
\noindent
\begin{tabular}{@{}p{8.00cm}p{3cm}@{}}
Wskaż nam Panie drogę po Bieszczadzkich szlakach & E - E/cis/H \\
Zagubionym – bądź echem w strumieniach i ptakach & A - A/H/E \\
I światłem w ciemności, jak twój księżyc blady & E - E/cis/H \\
Gdzie umilkły cerkwie i zdziczały sady & A - A/H/E
\end{tabular}

\end{document}
