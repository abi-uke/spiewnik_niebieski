%w Na drogi złe, dni zwyczajne
%r I będzie przebiegać muzyka
\documentclass[a5paper]{article}
 \usepackage[english,bulgarian,russian,ukrainian,polish]{babel}
 \usepackage[utf8]{inputenc}
 %\usepackage{polski}
 \usepackage[T1]{fontenc}
 \usepackage[margin=1.5cm]{geometry}
 \usepackage{multicol}
 \setlength\columnsep{10pt}
 \begin{document}
 %\pagenumbering{gobble}


\noindent
\fontsize{12pt}{15pt}\selectfont
\textbf{Ballada na złe drogi} \\
\fontsize{8pt}{10pt}\selectfont
EKT Gdynia, sł. W. Chiliński, muz. E. Adamiak \\ \\
\fontsize{10pt}{12pt}\selectfont
\leftskip0cm
\begin{tabular}{@{}p{8.5cm}p{3cm}@{}}
\noindent
& a F d E \\ 
Na drogi złe, dni zwyczajne & d E \\
I na najwyższe z progów & a F \\
Dostaliśmy w dłonie balladę & d E \\
I pachnie jak owoc głogu. & a A7 \\ \\
\end{tabular}

\leftskip1cm
\noindent
\begin{tabular}{@{}p{7.5cm}p{3cm}@{}} 
I będzie przebiegać muzyka & d G \\
Czy ty wiesz, jak to dużo po dniu & C F \\
I w wierszu nam będzie rozkwitać & d E \\
Ballada - posag mój. & a A7(a) \\ \\
\end{tabular}

\leftskip0cm
\noindent
\begin{tabular}{@{}p{8.5cm}p{3cm}@{}}
Na ludzi o szarych obliczach \\
Na ścieżki i wilcze doły. \\
Gdy zechce, na głos będzie krzyczeć \\
I w miejscu nam nie ustoi. \\ \\
\end{tabular}

\leftskip1cm
\noindent
\begin{tabular}{@{}p{8.5cm}p{3cm}@{}}
I będzie… \\ \\
\end{tabular}

\leftskip0cm
\noindent
\begin{tabular}{@{}p{8.5cm}p{3cm}@{}}
A kiedy będziemy odchodzić, \\
Hen - do Krainy Łowów \\
Błękitne się niebo otworzy \\
I spadnie jak owoc głogu. \\ \\
\end{tabular}

\leftskip1cm
\noindent
\begin{tabular}{@{}p{8.5cm}p{3cm}@{}}
I będzie…
\end{tabular}

\end{document}
