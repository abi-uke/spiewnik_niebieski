\documentclass[a5paper]{article}
 \usepackage[english,bulgarian,russian,ukrainian,polish]{babel}
 \usepackage[utf8]{inputenc}
 %\usepackage{polski}
 \usepackage[T1]{fontenc}
 \usepackage[margin=1.5cm]{geometry}
 \usepackage{multicol}
 \setlength\columnsep{10pt}
 \begin{document}
 %\pagenumbering{gobble}


\noindent
\fontsize{12pt}{15pt}\selectfont
\textbf{Jak} \\
\fontsize{8pt}{10pt}\selectfont
SDM, sł. E. Stachura, muz. K. Myszkowski \\ \\
\fontsize{10pt}{12pt}\selectfont
\leftskip0cm
\begin{tabular}{@{}p{9.00cm}p{3cm}@{}}
\noindent
Jak po nocnym niebie sunące białe obłoki nad lasem & D A G D \\
Jak na szyi wędrowca apaszka szamotana wiatrem & e G D \\
Jak wyciągnięte tam powyżej gwieździste ramiona wasze & A G D \\
A tu są nasze, a tu są nasze. & e G D \\

Jak suchy szloch w tę dżdżystą noc & A \\
Jak winny - i - niewinny sumienia wyrzut, & G D \\
Że się żyje, gdy umarło tylu, tylu, tylu. & e G D \\
Jak suchy szloch w tę dżdżystą noc & A \\
Jak lizać rany celnie zadane & G D \\
Jak lepić serce w proch potrzaskane & e G D \\
Jak suchy szloch w tę dżdżystą noc & A \\
Pudowy kamień, pudowy kamień & G D \\
Jak na nim stanę, on na mnie stanie & e G \\
On na mnie stanie, spod niego wstanę & D \\
Jak suchy szloch w tę dżdżystą noc & A \\
Jak złota kula nad wodami & G D \\
Jak świt pod spuchniętymi powiekami & e G D \\

Jak zorze miłe, śliczne polany & A \\
Jak słońca pierś, jak garb swój nieść & G D \\
Jak do was, siostry mgławicowe, ten zawodzący śpiew & e G D \\

Jak biec do końca, potem odpoczniesz, potem odpoczniesz & A G \\
Cudne manowce, cudne manowce, cudne, cudne manowce & D e G D
\end{tabular}

\end{document}
