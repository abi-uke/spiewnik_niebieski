%w Wystukaj po torach do mnie list
\documentclass[a5paper]{article}
 \usepackage[english,bulgarian,russian,ukrainian,polish]{babel}
 \usepackage[utf8]{inputenc}
 %\usepackage{polski}
 \usepackage[T1]{fontenc}
 \usepackage[margin=1.5cm]{geometry}
 \usepackage{multicol}
 \setlength\columnsep{10pt}
 \begin{document}
 %\pagenumbering{gobble}


\noindent
\fontsize{12pt}{15pt}\selectfont
\textbf{Blues dla Małej} \\
\fontsize{8pt}{10pt}\selectfont
Stare Dobre Małżeństwo, sł. A. Ziemianin, muz. K. Myszkowski \\ \\
\fontsize{10pt}{12pt}\selectfont
\leftskip0cm
\begin{tabular}{@{}p{8.5cm}p{3cm}@{}}
\noindent
Wystukaj po torach do mnie list & C G (lub h7/5-) \\
Wtedy naprawdę nie wyjedziesz cła & a G \\
Niech będzie w nim lokomotywy gwizd & F C \\
Tylko to zrób jeszcze dla mnie Mała & d (lub h7/5-) E7 a \\ \\
\end{tabular}

\leftskip0cm
\noindent
\begin{tabular}{@{}p{8.5cm}p{3cm}@{}}
Wystukaj po torach do mnie list & C G (lub h7/5-)\\
Choćby w alfabecie Morse'a & a e \\
Moja ulica jeszcze twardo śpi & F C \\
Jeśli tak chcesz - w liście zostań & d (lub h7/5-) E7 a \\ \\
\end{tabular}

\leftskip1cm
\noindent
\begin{tabular}{@{}p{7.5cm}p{3cm}@{}}
A mogliśmy Mała razem łąką iść & E (lub h7/5-)\\
Świt witać po kolana w rosie & a \\
A mogliśmy Mała razem piwo pić & G \\
Dom nasz zamienić na sto pociech & E \\
A mogliśmy Mała konie kraść & F \\
Z niebieskiego boskiego pastwiska & C \\
A mogliśmy Mała w środku lata & d (lub h7/5-)\\
Zbudować słoneczną przystań & E7 a \\ \\
\end{tabular}

\leftskip0cm
\noindent
\begin{tabular}{@{}p{9.5cm}@{}}
Napisz od serca do mnie list \\
I zamieszkaj w tym liście cała \\
Niech śmiechu dużo będzie w nim \\
Obiecaj mi to dzisiaj Mała \\ \\
\end{tabular}

\leftskip0cm
\noindent
\begin{tabular}{@{}p{9.5cm}@{}}
Napisz od serca do mnie list \\
Lecz nie wysyłaj go nigdy \\
W szufladzie zamknij go na klucz \\
Niech czeka wciąż lepszych dni
\end{tabular}

\end{document}
