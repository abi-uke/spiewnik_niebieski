%w Wolniej, wolniej wstrzymaj konia
\documentclass[a5paper]{article}
 \usepackage[english,bulgarian,russian,ukrainian,polish]{babel}
 \usepackage[utf8]{inputenc}
 %\usepackage{polski}
 \usepackage[T1]{fontenc}
 \usepackage[margin=1.5cm]{geometry}
 \usepackage{multicol}
 \setlength\columnsep{10pt}
 \begin{document}
 %\pagenumbering{gobble}


\noindent
\fontsize{12pt}{15pt}\selectfont
\textbf{Ballada o krzyżowcu} \\
\fontsize{8pt}{10pt}\selectfont
Mirosław Hrynkiewicz \\ \\
\fontsize{10pt}{12pt}\selectfont
\leftskip0cm
\begin{tabular}{@{}p{7.5cm}p{3cm}@{}}
\noindent
Wolniej, wolniej, wstrzymaj konia, & e \\
Dokąd pędzisz w stal odziany? & A \\
Pewnie tam, gdzie błyszczą w dali & C \\
Jeruzalem białe ściany? & D \\ \\

Pewnie myślisz, że w świątyni \\
Zniewolony pan twój czeka, \\
Żebyś przyszedł go ocalić, \\
Żebyś przybył doń z daleka. \\ \\

Wolniej, wolniej, wstrzymaj konia\\
Byłem dzisiaj w Jeruzalem.\\
Przemierzyłem puste sale,\\
Pana twego nie widziałem.\\\\

Pan opuścił święte miasto, \\
Przed minutą, przed godziną,\\
W chłodnym gaju na pustyni\\
Z Mahometem pije wino.\\\\

Wolniej, wolniej, wstrzymaj konia\\
Chcesz oblegać Jeruzalem?\\
Strzegą go wysokie wieże,\\
Strzegą go Mahometanie.\\\\

Pan opuścił święte miasto.\\
Na nic poświęcenie twoje.\\
Po cóż niszczyć białe ściany,\\
Po cóż ludzi niepokoić?\\\\

Wolniej, wolniej, wstrzymaj konia\\
Porzuć walkę niepotrzebną.\\
Porzuć miecz i włócznię swoją\\
I jedź ze mną, i jedź ze mną.\\\\

Bo gdy szlakiem ku północy\\
Podążają hufce ludne,\\
Ja podnoszę dumnie głowę\\
I odjeżdżam na południe.
\end{tabular}

\end{document}
