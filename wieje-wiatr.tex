\documentclass[a5paper]{article}
 \usepackage[english,bulgarian,russian,ukrainian,polish]{babel}
 \usepackage[utf8]{inputenc}
 %\usepackage{polski}
 \usepackage[T1]{fontenc}
 \usepackage[margin=1.5cm]{geometry}
 \usepackage{multicol}
 \setlength\columnsep{10pt}
 \begin{document}
 %\pagenumbering{gobble}


\noindent
\fontsize{12pt}{15pt}\selectfont
\textbf{Wieje wiatr} \\
\fontsize{8pt}{10pt}\selectfont
Dom o Zielonych Progach / sł. i muz. Wojtek Szymański \\ \\
\fontsize{10pt}{12pt}\selectfont
\leftskip0cm
\begin{tabular}{@{}p{7.50cm}p{3cm}@{}}
\noindent
/Wstęp: 7, 8 próg zaczynamy dwie struny E i H, \\
i w dół trzy pozycje, i dźwięk E/ \\ \\

Wieje wiatr w mojej głowie & e D \\
Nad doliny i nad sady niesie mnie & a7 h7 e \\
Wieje wiatr w mojej głowie \\
On pieśń wolności niesie mi \\ \\

I choćbym drogę zagubił	& C* \\
I nie wiedział dokąd iść & h* \\
I wszystko co wokół mnie \\
Straciło nagle sens \\
Przyjaciele by odeszli \\
Znikli nagle by gdzieś \\
I cała wiara moja \\
Rozpadła by się w pył \\ \\

I choćbym całą nadzieję \\ 
Utracił z dnia na dzień \\
A chmury czarne jak noc \\
Przyćmiły by mi słońce \\
I sam bym został jak palec \\
Pośród nieprzebytych gór \\
I miłości bym nie miał \\
Zagubił bym ją \\ \\
\emph{C* - bez palca na strunie H} \\
\emph{h* - bez bare}
\end{tabular}

\end{document}
