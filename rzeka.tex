%w Wsłuchany w twą cichą piosenkę
%r O dobra rzeko, o mądra wodo
\documentclass[a5paper]{article}
 \usepackage[english,bulgarian,russian,ukrainian,polish]{babel}
 \usepackage[utf8]{inputenc}
 %\usepackage{polski}
 \usepackage[T1]{fontenc}
 \usepackage[margin=1.5cm]{geometry}
 \usepackage{multicol}
 \setlength\columnsep{10pt}
 \begin{document}
 %\pagenumbering{gobble}


\noindent
\fontsize{12pt}{15pt}\selectfont
\textbf{Rzeka} \\
\fontsize{8pt}{10pt}\selectfont
Wolna Grupa Bukowina / słowa i muzyka: W. Jarociński \\ \\
\fontsize{10pt}{12pt}\selectfont
\leftskip0cm
\begin{tabular}{@{}p{7.50cm}p{3cm}@{}}
\noindent
E A2 E A2 \\ \\

Wsłuchany w twą cichą piosenkę & E A2  E A2 \\
Wyszedłem nad brzeg pierwszy raz.  & E A2  gis gis7 \\
Wiedziałem już rzeko, że kocham cię rzeko, & A gis7 cis \\
Że odtąd pójdę z tobą. & A gis fis H7 \\ \\
\end{tabular}

\leftskip1cm
\noindent
\begin{tabular}{@{}p{6.50cm}p{4cm}@{}}
O dobra rzeko, o mądra wodo, & E A2 E A2 E gis cis E7 \\
Wiedziałaś gdzie stopy znużone prowadzić, & A gis cis \\
Gdy sił już było brak.  & A gis fis7 H7 \\
Brak. & A2 \\
& E A2 E A2 E A2 \\ \\
\end{tabular}

\leftskip0cm
\noindent
\begin{tabular}{@{}p{8.00cm}p{3cm}@{}}
Wieże miast, łuny świateł, \\
Ich oczy Zszarzałe nie raz. \\
Witały mnie pustką, żegnały milczeniem \\
Gdym stał się twoim nurtem. \\
\end{tabular}

\leftskip1cm
\noindent
\begin{tabular}{@{}p{8.00cm}p{3cm}@{}}
	O dobra rzeko… \\ \\
\end{tabular}

\leftskip0cm
\noindent
\begin{tabular}{@{}p{8.00cm}p{3cm}@{}}
Po dziś dzień z tobą rzeko, \\
Gdzież począł, gdzie kres dał ci Bóg. \\
Ach, życia mi braknie, by szlak twój przemierzyć, \\
By poznać twą melodię. \\ \\
\end{tabular}

\leftskip1cm
\noindent
\begin{tabular}{@{}p{8.00cm}p{3cm}@{}}
O dobra rzeko…
\end{tabular}

\end{document}
