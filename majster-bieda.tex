%w Skąd przychodził, kto go znał
\documentclass[a5paper]{article}
 \usepackage[english,bulgarian,russian,ukrainian,polish]{babel}
 \usepackage[utf8]{inputenc}
 %\usepackage{polski}
 \usepackage[T1]{fontenc}
 \usepackage[margin=1.5cm]{geometry}
 \usepackage{multicol}
 \setlength\columnsep{10pt}
 \begin{document}
 %\pagenumbering{gobble}


\noindent
\fontsize{12pt}{15pt}\selectfont
\textbf{Majster bieda} \\
\fontsize{8pt}{10pt}\selectfont
Wolna Grupa Bukowina \\ \\
\fontsize{10pt}{12pt}\selectfont
\leftskip0cm
\begin{tabular}{@{}p{8.00cm}p{3cm}@{}}
\noindent
& C F e d G \\
Skąd przychodził, kto go znał & C F \\
Kto mu rękę podał kiedy & C d G \\
Nad rowem siadał, wyjmował chleb & C F \\
Serem przekładał i dzielił się z psem & e a \\
Tyle wszystkiego, co sobą miał & F e d G \\
Majster Bieda & C \\ \\
\end{tabular}

\leftskip0cm
\noindent
\begin{tabular}{@{}p{8.00cm}@{}}
Czapkę z głowy ściągał, gdy \\
Wiatr gałęzie chylił drzewom \\
Śmiał się do słońca i śpiewał do gwiazd \\
Drogą bez końca co przed nim szła \\
Znał jak pięć palców, jak szeląg zły \\
Majster Bieda \\ \\
\end{tabular}

\leftskip0cm
\noindent
\begin{tabular}{@{}p{8.00cm}@{}}
Nikt nie pytał skąd się wziął \\
Gdy do ognia się przysiadał \\
Wtulał się w krąg ciepła jak w kożuch \\
Zmęczony drogą wędrowiec boży \\
Zasypiał długo gapiąc się w noc \\
Majster Bieda \\ \\
\end{tabular}

\leftskip0cm
\noindent
\begin{tabular}{@{}p{8.00cm}p{3cm}@{}}
Aż nastąpił taki rok & \\
Smutny rok, tak widać trzeba & \\
Nie przyszedł Bieda zieloną wiosną & \\
Miejsce, gdzie siadał, zielskiem zarosło & \\
I choć niejeden wytężał wzrok & \\
Choć lato pustym gościńcem przeszło & \\
Z rudymi liśćmi jesienną schedą & \\
Wiatrem niesiony popłynął w przeszłość /x3 & \\
Majster Bieda
\end{tabular}

\end{document}
