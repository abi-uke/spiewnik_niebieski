%w Z brzękiem ostróg wjechałem do miasta
%r Usiadź razem ze mną, spróbuj mego wina
\documentclass[a5paper]{article}
 \usepackage[english,bulgarian,russian,ukrainian,polish]{babel}
 \usepackage[utf8]{inputenc}
 %\usepackage{polski}
 \usepackage[T1]{fontenc}
 \usepackage[margin=1.5cm]{geometry}
 \usepackage{multicol}
 \setlength\columnsep{10pt}
 \begin{document}
 %\pagenumbering{gobble}


\noindent
\fontsize{12pt}{15pt}\selectfont
\textbf{Jesienne wino} \\
\fontsize{8pt}{10pt}\selectfont
Andrzej Koczewski \\ \\
\fontsize{10pt}{12pt}\selectfont
\leftskip0cm
\begin{tabular}{@{}p{8.00cm}p{3cm}@{}}
\noindent
& e D x3 \\
Z brzękiem ostróg wjechałem do miasta, & e D e (D) \\
Pod jesień było, czas złotych liści nastał, & e D G (G) \\
W kieszeni worek srebra, czas do domu, & a G D e \\
Wtem za plecami woła głos. & D C e (D e D) \\ \\
\end{tabular}

\leftskip1cm
\noindent
\begin{tabular}{@{}p{7.00cm}p{3cm}@{}}
"Usiądź razem ze mną & e G \\
Spróbuj mego wina, & D G \\
Z czereśni, wiśni resztek lata, & a G \\
Choć jesień się zaczyna. & D C \\
Tyle tej jesieni & e G \\
Jeszcze jest przed nami, & D G \\
Zdążysz wrócić do domu & a G \\
Nim noc zawita nad drogami. Hej!" & D C e (D e D) \\ \\
\end{tabular}

\leftskip0cm
\noindent
\begin{tabular}{@{}p{8.00cm}p{3cm}@{}}
Słońce stało w zenicie, bił południowy żar, \\
A w gardle kurz przebytych dróg. \\
Co tam? Spocznę chwilę, przecież nie zaszkodzi, \\
Do przejścia niedaleką jeszcze drogę mam. \\
(a ona kusi) \\ \\
\end{tabular}

\leftskip1cm
\noindent
\begin{tabular}{@{}p{8.00cm}p{3cm}@{}}
"Usiądź… \\ \\
\end{tabular}

\leftskip0cm
\noindent
\begin{tabular}{@{}p{8.00cm}p{3cm}@{}}
Zbudziłem się w czerwieniach zachodu, \\
Pod stara karczmą, co rynek zamyka. \\
Zabrała moje srebro, duszę i ostrogi, \\
Zostało pragnienie i tępy głowy ból. \\
(i pamięć jej słów) \\ \\
\end{tabular}

\leftskip1cm
\noindent
\begin{tabular}{@{}p{8.00cm}p{3cm}@{}}
"Usiądź…
\end{tabular}

\end{document}
