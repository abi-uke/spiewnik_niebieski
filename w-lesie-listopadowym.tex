%r Wokół góry, góry i góry
\documentclass[a5paper]{article}
 \usepackage[english,bulgarian,russian,ukrainian,polish]{babel}
 \usepackage[utf8]{inputenc}
 %\usepackage{polski}
 \usepackage[T1]{fontenc}
 \usepackage[margin=1.5cm]{geometry}
 \usepackage{multicol}
 \setlength\columnsep{10pt}
 \begin{document}
 %\pagenumbering{gobble}


\noindent
\fontsize{12pt}{15pt}\selectfont
\textbf{W lesie listopadowym} \\
\fontsize{8pt}{10pt}\selectfont
sł. Jerzy Harasymowicz, muz. Andrzej Koczewski, Zbigniej Bogdański \\ \\
\fontsize{10pt}{12pt}\selectfont
\leftskip0cm
\begin{tabular}{@{}p{8.00cm}p{3cm}@{}}
\noindent
kapodaster II próg & \\ \\
\end{tabular}

\leftskip1cm
\noindent
\begin{tabular}{@{}p{7.00cm}p{3cm}@{}}
Wokół góry, góry i góry	& E7 a \\
I całe moje życie w górach & E7 a \\
Ileż piękniej drozdy leśne śpiewają & C G \\
Niż śpiewak płatny na chórach & E7 a \\
Wokół lasy, lasy i wiatr & E7 a \\
I całe życie w wiatru świstach & E7 a \\
Wszyscy których kocham wita Was	& C G \\
Modrzewia ikona złocista & E7 a \\ \\
\end{tabular}

\leftskip0cm
\noindent
\begin{tabular}{@{}p{8.00cm}p{3cm}@{}}
Jak łasiczki ścieżka w śniegach & d E7 a \\
Droga życia była kręta & d G C a \\
Teraz z lasów zeszła na mnie & d E7 a \\
Młodych jodeł zieleń święta & E7 a \\
Ważne są tylko kopuły pieśni & d E7 a \\
Które na górze wysokiej zostaną & d G C a \\
Nikt nie szuka inicjałów cieśli & d E7 a \\
Gdy cieśle dom postawią & E7 a \\ \\
\end{tabular}

\leftskip1cm
\noindent
\begin{tabular}{@{}p{7.00cm}p{3cm}@{}}
Wokół góry… \\ \\
\end{tabular}

\leftskip0cm
\noindent
\begin{tabular}{@{}p{8.00cm}@{}}
Nieludzką ręką malowany jest \\
Wielki smutek duszy mojej \\
Lecz nawet złockiej ikonie \\
Ja nigdy nic nie powiem \\
Przyjaciele, którzy jemiołę czcicie \\
Dobrze, że chodzicie światem \\
Wkrótce jodełkę zieloną spalicie \\ 
By darzyła was ciepłym latem \\ \\
\end{tabular}

\leftskip1cm
\noindent
\begin{tabular}{@{}p{7.00cm}@{}}
Wokół góry…
\end{tabular}

\end{document}
