%w Pierzaste brzyzgi światła przesiane
%r Wtulony w las Przegibek się gibie
\documentclass[a5paper]{article}
 \usepackage[english,bulgarian,russian,ukrainian,polish]{babel}
 \usepackage[utf8]{inputenc}
 %\usepackage{polski}
 \usepackage[T1]{fontenc}
 \usepackage[margin=1.5cm]{geometry}
 \usepackage{multicol}
 \setlength\columnsep{10pt}
 \begin{document}
 %\pagenumbering{gobble}


\noindent
\fontsize{12pt}{15pt}\selectfont
\textbf{Przegibek} \\
\fontsize{8pt}{10pt}\selectfont
Na Bani / sł.: Tom Borkowski, muz.: Bartłomiej Adamczak\\ \\
\fontsize{10pt}{12pt}\selectfont
\leftskip0cm
\begin{tabular}{@{}p{8.00cm}p{3cm}@{}}
\noindent
Pierzaste bryzgi światła przesiane & D h7 \\
Przez świerków niezłomne gałęzie & C A A4 \\
Trzymają nas w szachu od rana & D h7 \\
A wiatr nas w górach uwięził & C A A4 \\ \\

Ostatnie gorące spojrzenie słońca & D h7 C \\
Przelewa się jeszcze po szczytach & a A4 \\
A świerszcz obłąkany pobrzmiewa przez drzewa & D h7 C \\
Melodią niepospolitą & A G A D \\
& h7, C, A, A4 \\
& D, h7, C, A, A4 \\ \\
\end{tabular}

\leftskip1cm
\noindent
\begin{tabular}{@{}p{7.00cm}p{3cm}@{}}
Wtulony w las Przegibek się gibie & D h7 \\
Na obie strony przełęczy & g6 B4 \\
A słońce uchodzi za góry dla których & D D7+ D7 \\
Niejeden z nas dziś się zmęczył & B4 D \\
& h7, C, G \\
& D, h7, C, G \\
& D \\ \\
\end{tabular}

\leftskip0cm
\noindent
\begin{tabular}{@{}p{8.00cm}p{3cm}@{}}
Posłuchaj jak wołają przestrzenie \\
Posłuchaj co mówią góry \\
Odszukaj swoje gwiazdy na niebie \\
I śpiewom wiatru zawtóruj \\ \\

Traw zapachami daj się omamić \\
Ku Bani wyciągnij ręce \\
A ona odpowie tobie bez słowa \\
Szybciej zabije ci serce \\ \\
\end{tabular}

\leftskip1cm
\noindent
\begin{tabular}{@{}p{7.00cm}p{3cm}@{}}
Wtulony w las Przegibek się gibie \\
Na obie strony przełęczy \\
A słońce uchodzi za góry dla których \\
Znów jutro się trochę pomęczysz \\ \\
\end{tabular}

\leftskip2cm
\noindent
\begin{tabular}{@{}p{6.00cm}p{3cm}@{}}
Wtulony w las Przegibek się gibie & D h7 \\
Na obie strony przełęczy & C A A4 \\
A słońce uchodzi za góry dla których & D h7 C \\
Nie raz jeszcze zechcesz się zmęczyć & A4 G A D
\end{tabular}

\end{document}
