%w Dość wytocyli bań próżnych przed domy kalecy
%r Niechaj zalśni Bukowina w barwie malin
\documentclass[a5paper]{article}
 \usepackage[english,bulgarian,russian,ukrainian,polish]{babel}
 \usepackage[utf8]{inputenc}
 %\usepackage{polski}
 \usepackage[T1]{fontenc}
 \usepackage[margin=1.5cm]{geometry}
 \usepackage{multicol}
 \setlength\columnsep{10pt}
 \begin{document}
 %\pagenumbering{gobble}


\noindent
\fontsize{12pt}{15pt}\selectfont
\textbf{Bukowina 2} \\
\fontsize{8pt}{10pt}\selectfont
Wolna Grupa Bukowina, sł. i muz. W. Belon  \\ \\
\fontsize{10pt}{12pt}\selectfont
\leftskip0cm
\begin{tabular}{@{}p{8.5cm}p{3cm}@{}}
\noindent
Dość wytoczyli bań próżnych przed domy kalecy & C d F C \\
Żyją jak żyli - bezwolni głusi i ślepi & C d F C \\
Nie współczuj - szkoda łez i żalu & d G e \\
Bezbarwni są, bo chcą być szarzy & d G C e a \\
Ty wyżej, wyżej bądź i dalej & e F Fis G C \\
Niż ci co się wyzbyli marzeń & d G C \\ \\
\end{tabular}

\leftskip1cm
\noindent
\begin{tabular}{@{}p{7.5cm}p{3cm}@{}}
Niechaj zalśni Bukowina w barwie malin & C F G \\
Niechaj zabrzmi Bukowina w wiatru szumie & C F G \\
Dzień minął dzień minął - nadszedł wieczór & C d C \\
Świece gwiazd zapalił & F G \\
Siadł przy ogniu pieśń posłyszał i umilkł. & C d  F C \\ \\
\end{tabular}

\leftskip0cm
\noindent
\begin{tabular}{@{}p{8.5cm}p{3cm}@{}}
Po dniach zgiełkliwych, po nocach wyłożonych brukiem \\
W zastygłym szkliwie gwiazd neonowych próżno szukać \\
Tego, co tylko zielonością \\
Na palcach zaplecionych drzemie \\
Rozewrzyj dłonie mocniej, mocniej \\
Za kark chwyć słońce, sięgnij w niebo \\ \\
\end{tabular}

\leftskip1cm
\noindent
\begin{tabular}{@{}p{8.5cm}p{3cm}@{}}
Niechaj zalśni Bukowina w barwie malin… \\ \\
\end{tabular}

\leftskip0cm
\noindent
\begin{tabular}{@{}p{8.5cm}p{3cm}@{}}
Odnaleźć musisz - gdzie góry chmurom dłoń podają \\
Gdzie deszcz i susza gdzie lipce, październiki, maje \\
Stają się rokiem, węzłem życia \\
Twój dom bukowy zawieszony \\
U nieba pnia, kroplą żywicy \\
Błękitny, złoty i zielony \\ \\
\end{tabular}

\leftskip1cm
\noindent
\begin{tabular}{@{}p{8.5cm}p{3cm}@{}}
Niechaj zalśni Bukowina w barwie malin…
\end{tabular}

\end{document}
