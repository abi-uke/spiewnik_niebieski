%w Jest takie miejsce już od lat mi dobrze znane
%r To mój drugi, rodzinny własny dom
\documentclass[a5paper]{article}
 \usepackage[english,bulgarian,russian,ukrainian,polish]{babel}
 \usepackage[utf8]{inputenc}
 %\usepackage{polski}
 \usepackage[T1]{fontenc}
 \usepackage[margin=1.5cm,inner=2cm]{geometry}
 \usepackage{multicol}
 \setlength\columnsep{10pt}
 \begin{document}
 %\pagenumbering{gobble}


\noindent
\fontsize{12pt}{15pt}\selectfont
\textbf{Beskidzki dom} \\
\fontsize{8pt}{10pt}\selectfont
sł. i muz.: Jarosław „Timur” Gawryś \\ \\
\fontsize{10pt}{12pt}\selectfont
\leftskip0cm
\begin{tabular}{@{}p{9.50cm}p{3cm}@{}}
\noindent
Jest takie miejsce już od lat mi dobrze znane & e D C G \\
Gdzie wołanie ciszy, liści buków, wiosną przeplatane & a C C D \\
W progu gościem jestem, w środku gospodarzem & e D C G \\
Zamieszkuje tam wraz ze mną moc przedziwnych zdarzeń & a C C D \\\\
\end{tabular}

\leftskip1cm
\noindent
\begin{tabular}{@{}p{8.50cm}p{3cm}@{}}
To mój drugi, rodzinny własny dom & e D C G \\
Zbudowałem go przed laty wyobraźnią swą & a C C D \\
Powędruję babiogórskim pasmem gór & e D C G \\
By odnaleźć starą młodość zostawioną tu & a C C D \\ \\
\end{tabular}

\leftskip0cm
\noindent
\begin{tabular}{@{}p{11.50cm}p{3cm}@{}}
Zatroskany w nurcie różnych ważnych spraw \\
Wyszukuję w kalendarzu parę wolnych dat \\
I zataczam w myślach drogi mojej kres \\
Co mnie wiedzie w ukochane góry, tam, gdzie był i jest \\ \\
\end{tabular}

\leftskip1cm
\noindent
\begin{tabular}{@{}p{9.50cm}p{3cm}@{}}
To mój drugi, rodzinny własny dom… \\ \\
\end{tabular}

\leftskip2cm
\noindent
\begin{tabular}{@{}p{7.50cm}p{3cm}@{}}
Gdzie mój dom & G C \\
Rodzinny dom \\
Beskidzki dom
\end{tabular}

\end{document}
