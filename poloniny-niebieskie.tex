%w Gdy nie zostanie po mnie nic
\documentclass[a5paper]{article}
 \usepackage[english,bulgarian,russian,ukrainian,polish]{babel}
 \usepackage[utf8]{inputenc}
 %\usepackage{polski}
 \usepackage[T1]{fontenc}
 \usepackage[margin=1.5cm]{geometry}
 \usepackage{multicol}
 \setlength\columnsep{10pt}
 \begin{document}
 %\pagenumbering{gobble}


\noindent
% \setlength{\columnseprule}{0.1pt}
\noindent
\fontsize{12pt}{15pt}\selectfont
\textbf{Połoniny niebieskie} \\
\fontsize{8pt}{10pt}\selectfont
słowa: Marek Dutkiewicz, muzyka: Adam Drąg \\ \\
\fontsize{10pt}{12pt}\selectfont
\leftskip0cm
\begin{tabular}{@{}p{7.50cm}p{3cm}@{}}
\noindent
Gdy nie zostanie po mnie nic & C* F* C* F* \\
Oprócz pożółkłych fotografii, & C* F* C* F* G \\
\end{tabular}

\leftskip0.5cm
\begin{tabular}{@{}p{6.50cm}p{3cm}@{}}
Błękitny mnie przywita świt, & e F C G \\
W miejscu, co nie ma go na mapie. & C* F* \\ \\
\end{tabular}

\leftskip0cm
\noindent
\begin{tabular}{@{}p{6.50cm}p{3cm}@{}}
I kiedy sypną na mnie piach, & \\
Gdy mnie okryją cztery deski, & \\
\end{tabular}

\leftskip0.5cm
\begin{tabular}{@{}p{6.50cm}p{3cm}@{}}
To pójdę tam, gdzie wiedzie szlak - & \\
Na połoniny, na niebieskie. & \\ \\
\end{tabular}

\leftskip0cm
\noindent
\begin{tabular}{@{}p{6.50cm}p{3cm}@{}}
Podwiezie mnie błękitny wóz, & \\
Ciągnięty przez błękitne konie; & \\
\end{tabular}

\leftskip0.5cm
\begin{tabular}{@{}p{6.50cm}p{3cm}@{}}
Przez świat błękitny będzie wiózł, & \\
Aż zaniebieszczy w dali błonie. & \\ \\
\end{tabular}

\leftskip0cm
\noindent
\begin{tabular}{@{}p{6.50cm}p{3cm}@{}}
Od zmartwień wolny i od trosk  & \\
Pójdę wygrzewać się na trawie,  & \\
\end{tabular}

\leftskip0.5cm
\begin{tabular}{@{}p{6.50cm}p{3cm}@{}}
A czasem, gdy mi przyjdzie chęć, & \\
Z góry na ziemię się pogapię. & \\ \\
\end{tabular}
 
\leftskip0cm
\noindent
\begin{tabular}{@{}p{6.50cm}p{3cm}@{}}
Popatrzę, jak wśród smukłych malw  & \\
Wiatr w przedwieczornej ciszy kona;  & \\
\end{tabular}

\leftskip0.5cm
\begin{tabular}{@{}p{6.50cm}p{3cm}@{}}
	Trochę mi tylko będzie żal, & \\
	Że trawa u was tak zielona, & \\
	Trochę mi tylko będzie żal, & \\
	Że trawa u was tak zielona, & \\
	Że trawa u was tak zielona, & \\
	Że trawa u was tak zielona. & \\ \\
\end{tabular}

\noindent
\emph{C* - (E-e) 032013} \\
\emph{F* - (E-e) 003213}
\end{document}
