%w Wieczorem przy stołach w moim barze
%r Panie barman jeszcze nalej
\documentclass[a5paper]{article}
 \usepackage[english,bulgarian,russian,ukrainian,polish]{babel}
 \usepackage[utf8]{inputenc}
 %\usepackage{polski}
 \usepackage[T1]{fontenc}
 \usepackage[margin=1.5cm]{geometry}
 \usepackage{multicol}
 \setlength\columnsep{10pt}
 \begin{document}
 %\pagenumbering{gobble}


\noindent
\fontsize{12pt}{15pt}\selectfont
\textbf{Barman} \\
\fontsize{8pt}{10pt}\selectfont
Grzmiąca Półlitrówa, sł. i muz. Paweł Małolepszy \\ \\
\fontsize{10pt}{12pt}\selectfont
\leftskip0cm
\begin{tabular}{@{}p{9cm}p{3cm}@{}}
\noindent
& G D A D G D A D
Wieczorem, przy stołach w moim barze, tym na przeciwko kościoła & D A D A \\
Siadają ludzkie twarze tych, którym Bóg pomóc nie zdołał. & A A D D \\
Nad blatem błyszczącym jak czoło starego księżyca w pełni & G G D D \\
Na smutno i na wesoło spijają się frajerzy dzielni.  & A A D D \\

\end{tabular}

\leftskip1cm
\noindent
\begin{tabular}{@{}p{8cm}p{3cm}@{}}
Panie barman jeszcze nalej & G D \\
Nie za dużo, tak w sam raz & A h \\
Nam do nieba coraz dalej & G D \\
Widać los nie kocha nas. & A D \\ \\
\end{tabular}

\leftskip0cm
\noindent
\begin{tabular}{@{}p{8.5cm}p{3cm}@{}}
Przed sumieniem się tłumaczą, że w tej właśnie szklance whisky \\
Dziś na dnie wreszcie zobaczą swoją gwiazdę, której szukać tutaj przyszli. \\
I piją za szczęście, co przeszło gdzieś obok, nie zauważone. \\
Za nadzieję zuchwale tak grzeszną, za swe spracowane dłonie.\\\\

Nad ranem, gdy sprzątam w moim barze okruchy szkła - serca stłuczone, \\
Widzę odeszłe stąd twarze i słyszę ich toasty niespełnione. \\
Za miłość, którą czas zabił, za karty, co kiepsko nam idą. \\
Za tego, co nam błogosławił, za to życie, co tak nam obrzydło. \\\\
\end{tabular}

\leftskip1cm
\noindent
\begin{tabular}{@{}p{8cm}p{3cm}@{}}
Panie barman jeszcze nalej \\
- nie za dużo, tak w sam raz. \\
Tak by głowę mieć na karku, \\
lecz by pamięć trafił szlag.\\

\end{document}
