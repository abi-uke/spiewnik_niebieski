\documentclass[a5paper]{article}
 \usepackage[english,bulgarian,russian,ukrainian,polish]{babel}
 \usepackage[utf8]{inputenc}
 %\usepackage{polski}
 \usepackage[T1]{fontenc}
 \usepackage[margin=1.5cm]{geometry}
 \usepackage{multicol}
 \setlength\columnsep{10pt}
 \begin{document}
 %\pagenumbering{gobble}


\noindent
\fontsize{12pt}{15pt}\selectfont
\textbf{Kiedy góral umiera / Góralska opowieść} \\
\fontsize{8pt}{10pt}\selectfont
P. Kasperczyk \\ \\
\fontsize{10pt}{12pt}\selectfont
\leftskip0cm
\begin{tabular}{@{}p{9.5cm}p{3cm}@{}}
\noindent
Kiedy góral umiera, to góry z żalu sine & D D7 \\
Pochylają nad nim głowy jak nad swoim synem & G D \\
Las w oddali szumi mu odwieczną pieśń bukową & e G D \\
A on długo się sposobi przed najdalszą drogą & e G D \\ \\
\end{tabular}

\leftskip0cm
\noindent
\begin{tabular}{@{}p{9.5cm}@{}}
Kiedy góral umiera, to nikt za nim nie płacze \\
Cicho czeka aż kostucha w okno zakołacze \\
Oczy jeszcze raz podniesie wysoko do nieba \\
By pożegnać góry swe, by im coś zaśpiewać \\ \\
\end{tabular}

\leftskip1cm
\noindent
\begin{tabular}{@{}p{8.5cm}p{3cm}@{}}
Góry moje, wierchy moje, otwórzcie swe ramiona & D e \\
Niech na miękkim z mchu posłaniu cichuteńko skonam & G D \\
Ojcze mój, halny wietrze, powiej ku północy & D e \\
Ciepłą drżącą swoją ręką zamknij zgasłe oczy & G D \\ \\
\end{tabular}

\leftskip1cm
\noindent
\begin{tabular}{@{}p{8.5cm}p{3cm}@{}}
Bym mógł w ziemię wrosnąć & e \\
Strzelić potem do słońca smreczyną & G D \\
I na zawsze szumieć już nad moją dziedziną & e G D \\ \\
\end{tabular}

\leftskip0cm
\noindent
\begin{tabular}{@{}p{9.5cm}@{}}
Kiedy góral umiera, to dzwony mu nie grają \\
Cicho wspina się do bramy góralskiego raju \\
Tylko strumień po kamieniach żałobną nutę składa \\
Tylko nocka czarnooka górom opowiada \\ \\
\end{tabular}

\leftskip0cm
\noindent
\begin{tabular}{@{}p{9.5cm}@{}}
A kiedy góral już umrze, to nikt nie układa baśni \\
Tylko w niebie roziskrzonym mała gwiazdka zgaśnie \\
Ziemia twarda, szorstką ręką tuli go do siebie \\
By na zawsze mógł pozostać pod góralskim niebem \\ \\
\end{tabular}

\leftskip1cm
\noindent
\begin{tabular}{@{}p{8.5cm}@{}}
Góry moje…
\end{tabular}
\end{document}
