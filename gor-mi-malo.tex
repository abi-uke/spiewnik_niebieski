%w Drogi Mistrzu, Mistrzu mojej drogi
\documentclass[a5paper]{article}
 \usepackage[english,bulgarian,russian,ukrainian,polish]{babel}
 \usepackage[utf8]{inputenc}
 %\usepackage{polski}
 \usepackage[T1]{fontenc}
 \usepackage[margin=1.5cm]{geometry}
 \usepackage{multicol}
 \setlength\columnsep{10pt}
 \begin{document}
 %\pagenumbering{gobble}


\noindent
\fontsize{12pt}{15pt}\selectfont
\textbf{Gór mi mało} \\
\fontsize{8pt}{10pt}\selectfont
Dom o Zielonych Progach / sł. Tomasz Borkowski, muz. Wojtek Szymański \\ \\
\fontsize{10pt}{12pt}\selectfont
\leftskip0cm
\begin{tabular}{@{}p{8.50cm}p{3cm}@{}}
\noindent
C d G* GA G* | x2 & \\ \\

Drogi Mistrzu, Mistrzu mojej drogi & C G \\
Mistrzu Jerzy i mistrzu Wojciechu & d G \\
Przez was w górach schodziłem nogi & C G \\
Nie mogąc złapać oddechu & d G \\ \\

Gór, co stoją nigdy nie dogonię \\
Znikających punktów na mapie \\
Jakie miejsce nazwę swym domem \\
Jakim dotrę do niego szlakiem \\ \\
\end{tabular}

\leftskip1cm
\noindent
\begin{tabular}{@{}p{7.50cm}p{3cm}@{}}
Gór mi mało i trzeba mi więcej & C G \\
Żeby przetrwać od zimy do zimy & a e \\
Ktoś mnie skazał na wieczną wędrówkę & F C \\
Po śladach, które sam zostawiłem & d G \\ \\

Góry, góry i ciągle mi nie dość & \\
Skazanemu na gór dożywocie & \\
Świat na dobre mi zbieszczadział & \\
Szczyty wolnym mijają mnie krokiem & \\ \\

C d G* GA G* | x2 \\ \\
\end{tabular}

\leftskip0cm
\noindent
\begin{tabular}{@{}p{8.50cm}@{}}
Pańscy święci, święci bezpańscy \\
Święty Jerzy, Mikołaju, Michale \\
Starodawni gór świętych mieszkańcy \\
Imię wasze pieśniami wychwalam \\ \\

Gór, co stoją nigdy nie dogonię \\
Znikających punktów na mapie \\
I chaty, by nazwać ją swym domem \\
Do której żaden szlak by nie trafił \\ \\
\end{tabular}

\leftskip1cm
\noindent
\begin{tabular}{@{}p{7.50cm}@{}}
Gór mi mało i trzeba mi więcej…
\end{tabular}

\end{document}
