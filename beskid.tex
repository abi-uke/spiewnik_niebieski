%w A w Beskidzie rozzłocony buk
%r Beskidzie, malowany cerkiewny dach
\documentclass[a5paper]{article}
 \usepackage[english,bulgarian,russian,ukrainian,polish]{babel}
 \usepackage[utf8]{inputenc}
 %\usepackage{polski}
 \usepackage[T1]{fontenc}
 \usepackage[margin=1.5cm]{geometry}
 \usepackage{multicol}
 \setlength\columnsep{10pt}
 \begin{document}
 %\pagenumbering{gobble}


\noindent
\fontsize{12pt}{15pt}\selectfont
\textbf{Beskid} \\
\fontsize{8pt}{10pt}\selectfont
SETA / sł. i muz. Andrzej Wierzbicki \\ \\
\fontsize{10pt}{12pt}\selectfont
\leftskip0cm
\begin{tabular}{@{}p{8.5cm}p{3cm}@{}}
\noindent
A w Beskidzie rozzłocony buk & G C D G \\
A w Beskidzie rozzłocony buk & G C G D \\
Będę chodził bukowiną & C D \\
Z dłutem w ręku & G \\
By w dziewczęcych twarzach & C \\
Uśmiech rzeźbić & G \\
Niech nie płaczą już & C D \\
Niech się cieszą po kapliczkach & C D \\
Moich snów. & G \\ \\
\end{tabular}

\leftskip1cm
\noindent
\begin{tabular}{@{}p{7.5cm}p{3cm}@{}}
Beskidzie, malowany cerkiewny dach & G C D G \\
Beskidzie, zapach miodu w bukowych pniach & G C H7 e \\
Tutaj wracam, gdy ruda jesień & C D \\
Na przełęcze swój tobół niesie & G C \\
Słucham bicia dzwonów & G \\
W przedwieczorny czas & C D \\ \\
Beskidzie, malowany wiatrami dom & G C D G \\
Beskidzie, tutaj słowa inaczej brzmią & G C H7 e \\
Kiedy krzyczę w jesienną ciszę & C D \\
Kiedy wiatrem szeleszczą liście & G C \\
Kiedy wolność się tuli w ciepło moich rąk & G C D \\
Gdy jak źrebak się tuli do mych rąk. & C D G \\ \\
\end{tabular}

\leftskip0cm
\noindent
\begin{tabular}{@{}p{8.5cm}p{3cm}@{}}
A w Beskidzie zamyślony czas\\
A w Beskidzie zamyślony czas\\
Będę chodził z nim poddaszem gór\\
By zerwanych marzeń struny\\
Przywiązywać niespokojnym dłoniom drzew\\
Niech mi grają na rozstajach\\
Moich dróg.
\end{tabular}

\end{document}
