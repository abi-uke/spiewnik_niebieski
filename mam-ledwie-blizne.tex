%w Za stary jestem by uwierzyć w rewolucję
\documentclass[a5paper]{article}
 \usepackage[english,bulgarian,russian,ukrainian,polish]{babel}
 \usepackage[utf8]{inputenc}
 %\usepackage{polski}
 \usepackage[T1]{fontenc}
 \usepackage[margin=1.5cm]{geometry}
 \usepackage{multicol}
 \setlength\columnsep{10pt}
 \begin{document}
 %\pagenumbering{gobble}


\noindent
\fontsize{12pt}{15pt}\selectfont
\textbf{Mam ledwie bliznę} \\
\fontsize{8pt}{10pt}\selectfont
sł. i muz. Jaromir Nohavica, tłum. Antoni Muracki \\ \\
\fontsize{10pt}{12pt}\selectfont
\leftskip0cm
\begin{tabular}{@{}p{9.5cm}p{3cm}@{}}
\noindent
Za stary jestem, by uwierzyć w rewolucję & C \\
Kryję się w cieniu, by przed światem uciec & G \\
I nie smakuje mi podgrzewany ryż z torebek & a F \\
W kieszeni aspiryna, bo mogę być w potrzebie & C G \\
A przez igielne ucho, po mojemu, przejść nie zdołam & a \\
I gnam po cichu, by mnie nie dopadła wilcza sfora & C \\
A co z aniołem? - gdyby ktoś wiedzieć chciał - & d \\
Mam ledwie bliznę, bo przy mnie stał & C G C \\ \\
\end{tabular}

\noindent
\leftskip0cm
\begin{tabular}{@{}p{9.5cm}p{3cm}@{}}
\noindent
Popiół na marynarce, krawat pokryty sadzą \\
Moje niezgrabne palce z supłem sobie nie poradzą \\
Gdy czasem mi do płaczu - szybko łzy połykam \\
I tańczę w zgodnym rytmie - dopóki gra muzyka	\\
Mam w oczach kawał świata, wiele mi wpadło w ręce	\\
A gdy mą trąbią sławę - mam zawstydzone serce	\\
A co z aniołem? - gdyby ktoś wiedzieć chciał -	\\
Mam ledwie bliznę, bo przy mnie stał	\\ \\
\end{tabular}

\noindent
\leftskip0cm
\begin{tabular}{@{}p{9.5cm}p{3cm}@{}}
\noindent
Poznałem prezydentów i drabów, co chcą zabić	\\
Nagi przyszedłem na świat i odejdę nagi	\\
Gdy miałem lat piętnaście – widziałem ruskie tanki	\\
Dzisiaj po przepowiednie odsyłam do cyganki	\\
Lecz zanim, święty Piotrze, wezwiesz na rozmowę	\\
Przed poezją czeską pozwól schylić głowę	\\
A co z aniołem? - gdyby ktoś wiedzieć chciał -	\\
Mam ledwie bliznę, bo przy mnie stał	\\ \\
\end{tabular}

\noindent
\leftskip0cm
\begin{tabular}{@{}p{9.5cm}p{3cm}@{}}
\noindent
W Paryżu l'Humanite po rosyjsku czytam z rana	\\
Z Biblii rozumiem zaś pojedyncze zdania	\\
Od plastikowych łyżek w Stanach miałem już zajady	\\
Do Hypernovej jadę, by się napić dobrej kawy	\\
Choć asa mam w zanadrzu, to mówię pas świadomie	\\
I chcę by kiedyś Banik dokopał Barcelonie	\\
A co z aniołem? - gdyby ktoś wiedzieć chciał -\\
Mam ledwie bliznę, bo przy mnie stał	\\ \\
\end{tabular}

\noindent
\leftskip0cm
\begin{tabular}{@{}p{9.5cm}p{3cm}@{}}
\noindent
Niektórzy mają skłonność do dziwnych upodobań	 \\
A ja cię, moja miła, tak niezmiennie kocham	 \\
Kiedy się krzątasz w kuchni, gotując konfitury	\\
Gdy dyrygujesz domem z rodzinnej partytury	\\
I choć nas w różne strony niesie spraw lawina	\\
To złe wspomnienia dla dobrych zapominam	\\
A co z aniołem? - gdyby ktoś wiedzieć chciał -	\\
Mam ledwie bliznę, bo przy mnie stał \\ \\
\end{tabular}

\end{document}
