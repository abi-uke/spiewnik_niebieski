%w Brzęk muchy w pustym dzbanie co stoi na półce
%r A jak dumnie się modrzy u ciszy podnóża
\documentclass[a5paper]{article}
 \usepackage[english,bulgarian,russian,ukrainian,polish]{babel}
 \usepackage[utf8]{inputenc}
 %\usepackage{polski}
 \usepackage[T1]{fontenc}
 \usepackage[margin=1.5cm]{geometry}
 \usepackage{multicol}
 \setlength\columnsep{10pt}
 \begin{document}
 %\pagenumbering{gobble}


\noindent
\fontsize{12pt}{15pt}\selectfont
\textbf{Zwiewność} \\
\fontsize{8pt}{10pt}\selectfont
Bez Jacka / sł. B.Leśmian, muz. Z. Stefański \\ \\
\fontsize{10pt}{12pt}\selectfont
\leftskip0cm
\begin{tabular}{@{}p{9.00cm}@{}}
\noindent
a G F E7 x2 \\ \\
\end{tabular}

\leftskip0cm
\noindent
\begin{tabular}{@{}p{9.00cm}p{3cm}@{}}
Brzęk muchy w pustym dzbanie, co stoi na półce & a \\
Smuga w oczach po znikłej za oknem jaskółce & G \\
Cień ręki na murawie, a wszystko niczyje & F \\
Ledwo się zazieleni, już ufa że żyje & E E7 \\ \\
\end{tabular}

\leftskip1cm
\noindent
\begin{tabular}{@{}p{8.00cm}p{3cm}@{}}
A jak dumnie się modrzy u ciszy podnóża & a \\
Jak buńczucznie do boju z mgłą się napurpurza & G \\
A jest go tak niewiele, że mniej niż niebiesko & F \\
Nic prócz tła, biały obłok z czerwoną przekreską & E E7 \\ \\
\end{tabular}

\leftskip0cm
\noindent
\begin{tabular}{@{}p{9.00cm}@{}}
Dal świata w ślepiach wróbla spotkanie traw z ciałem \\
Szmery w studni, ja w lesie, byłeś mgłą – bywałem \\
Usta twoje w alei, świt pod groblą, w młynie \\
Słońce w bramie na oścież, zgon pszczół w koniczynie \\ \\
\end{tabular}

\leftskip1cm
\noindent
\begin{tabular}{@{}p{8.00cm}@{}}
A jak dumnie się modrzy… \\ \\
\end{tabular}

\leftskip0cm
\noindent
\begin{tabular}{@{}p{9.00cm}@{}}
a G F E7 x2 \\ \\
\end{tabular}

\leftskip0cm
\noindent
\begin{tabular}{@{}p{9.00cm}@{}}
Chód po ziemi człowieka, co na widnokresie \\
Malejąc mało zwiewną gęstwę ciała niesie \\
I w tej gęstwie się modli i gmatwa co chwila \\
I wyziera z gęstwy w świat i na motyla \\ \\
\end{tabular}

\leftskip1cm
\noindent
\begin{tabular}{@{}p{8.00cm}@{}}
A jak dumnie się modrzy…
\end{tabular}

\end{document}
