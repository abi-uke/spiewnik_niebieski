%w Gdy cię pierwszy raz ujrzałem
%r Lato nam umyka po strumykach
\documentclass[a5paper]{article}
 \usepackage[english,bulgarian,russian,ukrainian,polish]{babel}
 \usepackage[utf8]{inputenc}
 %\usepackage{polski}
 \usepackage[T1]{fontenc}
 \usepackage[margin=1.5cm]{geometry}
 \usepackage{multicol}
 \setlength\columnsep{10pt}
 \begin{document}
 %\pagenumbering{gobble}


\noindent
\fontsize{12pt}{15pt}\selectfont
\textbf{Lato} \\
\fontsize{8pt}{10pt}\selectfont
Słodki Całus od Buby \\ \\
\fontsize{10pt}{12pt}\selectfont
\leftskip0cm
\begin{tabular}{@{}p{7.50cm}p{3cm}@{}}
\noindent
Gdy cię pierwszy raz ujrzałem & C/C G4/H \\
Jasno zrozumiałem, że & a7/A a7/G \\
Muszę bronić się przed tobą, & F \\
Na najmniejszy zważać gest. & F C \\
Wypalony letnim słońcem \\
Zniknął wątpliwości ślad: \\
Zamieniłem „czy” i „kiedy”, \\
Na pytanie „gdzie” i „jak”? \\ \\
\end{tabular}

\leftskip1cm
\noindent
\begin{tabular}{@{}p{6.50cm}p{3cm}@{}}
Lato nam umyka po strumykach, & C/C G4/H a7/A \\
Po zielonych stokach wzgórz, & a7/G C \\
O wczoraj się nie pyta- & C/C G4/H \\
Nie pamięta już… & a7/A e \\
Lato ciemną, pełną szeptów nocą & C/C G4/H a7/A \\
Kusi, jak & a7/G \\
Na twoich opalonych plecach	& F \\
Jasny po ramiączku ślad. & F C \\ \\
\end{tabular}

\leftskip0cm
\noindent
\begin{tabular}{@{}p{7.50cm}p{3cm}@{}}
Zapukałem w twoje okno, \\
Powiedziałaś „wejdź”- \\
Z gęstej poziomkowej nocy \\
Jasny świt wynurzył się. \\
Zapukałem do twojego serca, \\
Powiedziałaś „idź”- \\
Na rozgrzaną ścieżkę sierpnia \\
Upadł pierwszy żółty liść. \\ \\
\end{tabular}

\leftskip1cm
\noindent
\begin{tabular}{@{}p{7.50cm}p{3cm}@{}}
Lato nam umyka… \\ \\
\end{tabular}

\leftskip0cm
\noindent
\begin{tabular}{@{}p{7.50cm}p{3cm}@{}}
Ciepłe noce pod gwiazdami, \\
Duszne pod namiotem sny, \\
Jesień stuka sztalugami, \\
Świat gruntuje bielą mgły. \\
Pierwszy świt w koronkach szronu, \\
Pełna słodkich jagód garść… \\
Jeśli tylko się przyśniłaś, \\
Czemu teraz tak mi żal? \\ \\
\end{tabular}

\leftskip1cm
\noindent
\begin{tabular}{@{}p{7.50cm}p{3cm}@{}}
Lato nam umyka… \\
\end{tabular}

\end{document}
