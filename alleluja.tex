%w Tajemny akord kiedyś brzmiał
\documentclass[a5paper]{article}
 \usepackage[english,bulgarian,russian,ukrainian,polish]{babel}
 \usepackage[utf8]{inputenc}
 %\usepackage{polski}
 \usepackage[T1]{fontenc}
 \usepackage[margin=1.5cm]{geometry}
 \usepackage{multicol}
 \setlength\columnsep{10pt}
 \begin{document}
 %\pagenumbering{gobble}


\noindent
\fontsize{12pt}{15pt}\selectfont
\textbf{Alleluja} \\
\fontsize{8pt}{10pt}\selectfont
sł. i muz. Leonard Cohen, tłum. Maciej Zembaty \\ \\
\fontsize{10pt}{12pt}\selectfont
\leftskip0cm
\begin{tabular}{@{}p{8.5cm}p{3cm}@{}}
\noindent
Tajemny akord kiedyś brzmiał, & C a \\
Pan cieszył się, gdy Dawid grał, & C a\\
Ale muzyki nikt dziś tak nie czuje. & F G C - G \\
Kwarta i kwinta tak to szło, & C E\\
Raz wyżej w dur, raz niżej w moll, & a F \\
Nieszczęsny król ułożył: Alleluja & G E a\\ \\
\end{tabular}

\leftskip1cm
\noindent
\begin{tabular}{@{}p{7.5cm}p{3cm}@{}}
Alleluja, alleluja, alleluja, alleluja & F C F C G C\\\\
\end{tabular}

\leftskip0cm
\noindent
\begin{tabular}{@{}p{8.5cm}p{3cm}@{}}
Na wiarę nic nie chciałeś brać.\\
Lecz sprawił to księżyca blask,\\
Że piękność jej na zawsze cię podbiła.\\
Kuchenne krzesło tronem twym,\\
Ostrzygła cię, już nie masz sił\\
I z gardła ci wydarła: Alleluja!\\\\
\end{tabular}

\leftskip1cm
\noindent
\begin{tabular}{@{}p{8.5cm}p{3cm}@{}}
Alleluja, alleluja, alleluja, alleluja\\\\
\end{tabular}

\leftskip0cm
\noindent
\begin{tabular}{@{}p{8.5cm}p{3cm}@{}}
Dlaczego mi zarzucasz wciąż,\\
Że nadaremno wzywam Go?\\
Ja przecież nawet nie znam Go z imienia.\\
Jest w każdym słowie światła błysk,\\
Nieważne, czy usłyszysz dziś\\
Najświętsze, czy nieczyste: Alleluja!\\\\
\end{tabular}

\leftskip1cm
\noindent
\begin{tabular}{@{}p{8.5cm}p{3cm}@{}}
Alleluja, alleluja, alleluja, alleluja\\\\
\end{tabular}

\leftskip0cm
\noindent
\begin{tabular}{@{}p{8.5cm}p{3cm}@{}}
Tak się starałem, ale cóż\\
Dotykam tylko, zamiast czuć.\\
Lecz mówię prawdę, nie chcę was oszukać.\\
I chociaż wszystko poszło źle, \\
Przed Panem Pieśni stawię się, \\
Śpiewając tylko jedno: Alleluja
\end{tabular}

\end{document}
