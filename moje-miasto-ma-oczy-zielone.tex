%w Chłodne parki aż kipią zielenią
\documentclass[a5paper]{article}
 \usepackage[english,bulgarian,russian,ukrainian,polish]{babel}
 \usepackage[utf8]{inputenc}
 %\usepackage{polski}
 \usepackage[T1]{fontenc}
 \usepackage[margin=1.5cm]{geometry}
 \usepackage{multicol}
 \setlength\columnsep{10pt}
 \begin{document}
 %\pagenumbering{gobble}


\noindent
\fontsize{12pt}{15pt}\selectfont
\textbf{Moje miasto ma oczy zielone} \\
\fontsize{8pt}{10pt}\selectfont
sł. i muz.: Krzysztof Jurkiewicz \\ \\
\fontsize{10pt}{12pt}\selectfont
\leftskip0cm
\begin{tabular}{@{}p{8.50cm}p{3cm}@{}}
\noindent
Chłodne parki aż kipią zielenią & C \\
Cieniem idziesz tak piękna i bosa & F G C \\
Skwarny lipiec spogląda na nasze spotkanie & G C F \\
Patrzy na złoty piasek we włosach & C G \\
Patrzy na złoty piasek we włosach & a G F \\
I jak jesteś słońcem spalona & a G F / G \\
A ty musisz koniecznie raz jeszcze spróbować & C G C \\
Czy Zatoka naprawdę jest słona & C G \\
& (C GC) \\ \\
Wiatr się ściga z żaglami po falach \\
Górą białe przegania obłoki \\
A wieczorem Starówka swoje bramy otwiera\\
Czeka na chłodny oddech Zatoki\\
Czeka na chłodny oddech Zatoki\\
I na wszystko co dzisiaj mi powiesz\\
Na twój taniec szalony i na pocałunki\\
W zakamarkach ukradkiem kradzione\\ \\

Kiedy spojrzysz ze wzgórza na dachy\\
Pierwszym blaskiem poranka złocone\\
Wtedy jak ja zobaczysz, jak ja uwierzysz że\\
Moje miasto ma oczy zielone\\
Moje miasto ma oczy zielone\\
W moim mieście się można zakochać\\
Gdy zalotnie zaplata między nocą a dniem\\
Złotą wstążkę plaży we włosach
\end{tabular}

\end{document}
