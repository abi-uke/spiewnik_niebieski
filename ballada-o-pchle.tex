%w To była miłość co zdarza się raz
%r Pchła z syfilisem
\documentclass[a5paper]{article}
 \usepackage[english,bulgarian,russian,ukrainian,polish]{babel}
 \usepackage[utf8]{inputenc}
 %\usepackage{polski}
 \usepackage[T1]{fontenc}
 \usepackage[margin=1.5cm]{geometry}
 \usepackage{multicol}
 \setlength\columnsep{10pt}
 \begin{document}
 %\pagenumbering{gobble}


\noindent
\fontsize{12pt}{15pt}\selectfont
\textbf{Ballada o pchle} \\
\fontsize{8pt}{10pt}\selectfont
Martin Lechowicz \\ \\
\fontsize{10pt}{12pt}\selectfont
\leftskip0cm
\begin{tabular}{@{}p{8.5cm}p{3cm}@{}}
\noindent
To była miłość co zdarza się raz & a e \\
Miłość co spokój skradła. & C D G e \\
Żyła raz pchła, co kochała psa, & a e \\
Psa, którego jadła. & C D e \\ \\

Jej oczy są zawsze pełne łez, \\
A serce z żalu usycha. \\
Bo nie dość, że ssie krew miłości swej \\
To jeszcze przenosi syfa. \\ \\
\end{tabular}

\leftskip1cm
\noindent
\begin{tabular}{@{}p{7.5cm}p{3cm}@{}}
Pchła z syfilisem, & e \\
Pchła z syfilisem, & D \\
Pchła z syfem, zakochana pchła. & a e \\

Pchła z syfilisem, & e \\
Pchła z syfilisem, & D \\
Pchła z syfem, nieszczęśliwa pchła. & a e D e \\ \\
\end{tabular}

\leftskip0cm
\noindent
\begin{tabular}{@{}p{8.5cm}p{3cm}@{}}
I z każdym łykiem kochanka krwi, \\
Żal pchle serce ściska i dusi. \\
Bo pragnie z nim spędzać swe wszystkie dni \\
A coś przecież k***a żreć musi. \\ \\

I krwawi wciąż ukochany pies, \\
Pchła jęczy, zawodzi i szlocha. \\
Bo nie wie, że w życiu już tak to jest \\
Że gryzie się to, co się kocha. \\ \\
\end{tabular}

\leftskip1cm
\noindent
\begin{tabular}{@{}p{8.5cm}p{3cm}@{}}
Pchła z syfilisem… \\ \\
\end{tabular}

\leftskip0cm
\noindent
\begin{tabular}{@{}p{8.5cm}p{3cm}@{}}
Przeklina pchła swój okrutny los, \\
I żądza zemsty w niej skrzy się. \\
Za jej cierpienia niech cały świat \\
Zatonie w syfilisie. \\ \\

Uważaj więc zanim pogłaskasz psa, \\
Choć piękny i sierść ma śliczną. \\
Bo mściwa pchła może obdarzyć cię \\
Chorobą weneryczną.\\ \\
\end{tabular}

\leftskip1cm
\noindent
\begin{tabular}{@{}p{8.5cm}p{3cm}@{}}
Pchła z syfilisem…
\end{tabular}

\end{document}
