%w Jak do groźnej rzeki
%r Miło, bracia, miło
\documentclass[a5paper]{article}
 \usepackage[english,bulgarian,russian,ukrainian,polish]{babel}
 \usepackage[utf8]{inputenc}
 %\usepackage{polski}
 \usepackage[T1]{fontenc}
 \usepackage[margin=1.5cm]{geometry}
 \usepackage{multicol}
 \setlength\columnsep{10pt}
 \begin{document}
 %\pagenumbering{gobble}


\noindent
\fontsize{12pt}{15pt}\selectfont
\textbf{Ataman} \\
\fontsize{8pt}{10pt}\selectfont
Lubelska Federacja Bardów \\ \\
\fontsize{10pt}{12pt}\selectfont
\leftskip0cm
\begin{tabular}{@{}p{9cm}p{3cm}@{}}
\noindent
Jak do groźnej rzeki, precz kozacy pchnęli, & a d E7 a \\
Precz kozacy pchnęli sto tysięcy koni w cwał. & G G7 C E7 \\
I pokryły pola i pokryły brzegi, & a d G C \\
Setki porąbanych i poprzestrzelanych ciał. & G G7 C E7 \\ \\
\end{tabular}

\leftskip2cm
\noindent
\begin{tabular}{@{}p{7cm}p{3cm}@{}}
Miło, bracia, miło & a d \\
Miło, bracia, żyć & G C \\
Z naszym atamanem & G G4 G7 \\
nie ma co się martwić nic & C E7 \\ \\
\end{tabular}

\leftskip1cm
\noindent
\begin{tabular}{@{}p{8.5cm}p{3cm}@{}}
Ataman wie dobrze i wybiera mądrze.\\
Szwadronami w konie, zapomnieli o mnie wnet\\
mają przydział doli, tej kozackiej woli\\
mnie pylista ziemia, rozpalony gorzki step.\\\\
\end{tabular}

\leftskip0cm
\noindent
\begin{tabular}{@{}p{8.5cm}p{3cm}@{}}
Ech, ta pierwsza kula, ech, ta pierwsza kula,\\
Ech  ta pierwsza kula, koń się o nią potknął sam… \\
Ech, ta druga kula, ech ta druga kula… \\
Ech, ta druga kula, co ją w sercu nosić mam. \\ \\
\end{tabular}

\leftskip1cm
\noindent
\begin{tabular}{@{}p{8.5cm}p{3cm}@{}}
Żonka się posmuci, za koleżką rzuci,\\
za innego wyjdzie, pamięć o mnie przyćmi dal\\
tej wolności szkoda gdy, już się po niej rzeka szkli,\\
szkoda ostrej szabli, bułanego konia żal…
\end{tabular}

\end{document}
