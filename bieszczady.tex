%w Tu w dolinach wstaje mgłą wilgotny dzień
%r Cicho potok gada, gwarzy pośród skał
\documentclass[a5paper]{article}
 \usepackage[english,bulgarian,russian,ukrainian,polish]{babel}
 \usepackage[utf8]{inputenc}
 %\usepackage{polski}
 \usepackage[T1]{fontenc}
 \usepackage[margin=1.5cm]{geometry}
 \usepackage{multicol}
 \setlength\columnsep{10pt}
 \begin{document}
 %\pagenumbering{gobble}


\noindent
\fontsize{12pt}{15pt}\selectfont
\textbf{Bieszczady} \\
\fontsize{8pt}{10pt}\selectfont
Andrzej Starzec, Maciej Płoński \\ \\
\fontsize{10pt}{12pt}\selectfont
\leftskip0cm
\begin{tabular}{@{}p{8.5cm}p{3cm}@{}}
\noindent
Tu w dolinach wstaje mgłą wilgotny dzień & e a \\
Szczyty ogniem płoną, stoki kryje cień & D7 G H7 \\
Mokre rosą trawy wypatrują dnia & e a \\
Ciepła, które pierwszy słońca promień da & D7 G H7 \\ \\
\end{tabular}

\leftskip1cm
\noindent
\begin{tabular}{@{}p{7.5cm}p{3cm}@{}}
Cicho potok gada, gwarzy pośród skał & G C D7 G \\
O tym deszczu, co z chmury trochę wody dał \\
Świerki zapatrzone w horyzontu kres \\
Głowy pragną wysoko, jak najwyżej wznieść \\ \\
\end{tabular}

\leftskip0cm
\noindent
\begin{tabular}{@{}p{9.5cm}p{3cm}@{}}
Tęczą kwiatów barwny połoniny łan \\
Słońcem wypełniony jagodowy dzban \\
Pachnie świeżym sianem pokos pysznych traw \\
Owiec dzwoneczkami cisza niebu gra \\ \\
\end{tabular}

\leftskip1cm
\noindent
\begin{tabular}{@{}p{8.5cm}p{3cm}@{}}
Cicho potok gada, gwarzy pośród skał… \\ \\
\end{tabular}

\leftskip0cm
\noindent
\begin{tabular}{@{}p{9.5cm}p{3cm}@{}} 
Serenadą świerszczy kaskadami gwiazd \\
Noc w zadumie kroczy mroku ścieląc płaszcz \\
Wielkim Wozem księżyc rusza na swój szlak \\
Pozłocistym sierpem gasi lampy dnia \\ \\
\end{tabular}

\leftskip1cm
\noindent
\begin{tabular}{@{}p{8.5cm}p{3cm}@{}} 
Cicho potok gada, gwarzy pośród skał…\\ \\
\end{tabular}

\leftskip0cm
\noindent
\begin{tabular}{@{}p{9.5cm}p{3cm}@{}}
W czwartek w Południku Styków dużo jest \\
Tosty i herbata, to posiłek fest  \\
Bruszek już wesoły, stoją kubki dwa \\
Śpiewać już możemy, na bok smutki dnia! \\ \\
\end{tabular}

\leftskip1cm
\noindent
\begin{tabular}{@{}p{8.5cm}p{3cm}@{}}
Cicho potok gada, gwarzy pośród skał…
\end{tabular}
\end{document}
