%w Popękana na nim zbroja
%r A iść przecież miałem w zupełnie inną stronę
\documentclass[a5paper]{article}
 \usepackage[english,bulgarian,russian,ukrainian,polish]{babel}
 \usepackage[utf8]{inputenc}
 %\usepackage{polski}
 \usepackage[T1]{fontenc}
 \usepackage[margin=1.5cm]{geometry}
 \usepackage{multicol}
 \setlength\columnsep{10pt}
 \begin{document}
 %\pagenumbering{gobble}


\noindent
\fontsize{12pt}{15pt}\selectfont
\textbf{Rycerz} \\
\fontsize{8pt}{10pt}\selectfont
Bez Jacka / sł. Z. Stefański, muz. Z. Stefański, J.H. Chrząstek \\ \\
\fontsize{10pt}{12pt}\selectfont
\leftskip0cm
\begin{tabular}{@{}p{8.50cm}p{3cm}@{}}
\noindent
Popękana na nim zbroja & e a \\
i zadano wiele ran, & C h \\
ledwie miecza pół w mdlejącej & G D \\
trzyma dłoni… & h \\
Kurz kolory skrył, a z proporca tylko strzęp & C h e C \\
swym łopotem krzyczy, że nie złoży obroni. & e D e \\
W koło wrogi tłum, zwartym kręgiem prze & e C h C h \\
i szyderstwem ciska w dumę tego trwania. & E D h \\
Tarcza pękła już u stóp leży w prochu hełm, & C h e C \\
Tylko oczu blask od ciosów go osłania. & E D e \\ \\
\end{tabular}

\leftskip1cm
\noindent
\begin{tabular}{@{}p{7.50cm}p{3cm}@{}}
A iść przecież miał w zupełnie inną stronę & C G D \\
i marzenia dłonią gładzić, a nie gniew. & C G D \\
Taki dziwny świat, wszędzie w koło tylko zew, & C h e C \\
a my wciąż w półdrogi tej, nieokreślonej. & e h e \\ \\
\end{tabular}

\leftskip0cm
\noindent
\begin{tabular}{@{}p{8.50cm}@{}}
Zobaczyłem raz wyraźnie, \\
jak ucieka moje dno, \\
a ja razem z nim zanurzam się w otchłanie, \\
kłamstwa lepki brud, \\
strachem był wolności głód, \\
ludzie prawi zostawali mi nieznani. \\ \\
\end{tabular}

\leftskip1cm
\noindent
\begin{tabular}{@{}p{7.50cm}p{3cm}@{}}
A iść przecież miałem w całkiem inną stronę…
\end{tabular}

\end{document}
