%w Tak smutno patrzysz przez otwarte okno
%r Usiądź tu obok mnie w fotelu
\documentclass[a5paper]{article}
 \usepackage[english,bulgarian,russian,ukrainian,polish]{babel}
 \usepackage[utf8]{inputenc}
 %\usepackage{polski}
 \usepackage[T1]{fontenc}
 \usepackage[margin=1.5cm]{geometry}
 \usepackage{multicol}
 \setlength\columnsep{10pt}
 \begin{document}
 %\pagenumbering{gobble}


\noindent
\fontsize{12pt}{15pt}\selectfont
\textbf{Nie chodź tam} \\
\fontsize{8pt}{10pt}\selectfont
Browar Żywiec \\ \\
\fontsize{10pt}{12pt}\selectfont
\leftskip0cm
\begin{tabular}{@{}p{8.50cm}p{3cm}@{}}
\noindent
Tak smutno patrzysz przez otwarte okno & a e \\
Już się latarnie w mieście zapalają	& F7+ C \\
I nic nie mówisz, patrzysz tylko w dół & F7+ F7+ D \\
Nie chodź tam, gdzie uliczny obcy tłum & a F G C \\ \\
\end{tabular}

\leftskip1cm
\noindent
\begin{tabular}{@{}p{7.50cm}p{3cm}@{}}
Usiądź tu obok mnie w fotelu & F7+ G C \\
Słonecznikowe pestki gryź & F7+ G C \\
A ja piosenkę Ci zanucę & F E a F \\
Tę, której słuchać chciałabyś & G C \\ \\
 
Nie strącaj z biurka kałamarza \\
Nie zrywaj z kalendarza dni \\
Jeśli nie lubisz tej piosenki \\
Inną, ładniejszą zagram Ci & C G C \\ \\
\end{tabular}

\leftskip0cm
\noindent
\begin{tabular}{@{}p{8.50cm}p{3cm}@{}} 
Zasnęło słońce gdzieś za antenami \\
Wyłażą koty z piwnic i śmietników \\
Zadzwonił tramwaj – już ostatni kurs \\
Nie chodź tam, puste place, zniknął tłum \\ \\
\end{tabular}

\leftskip1cm
\noindent
\begin{tabular}{@{}p{7.50cm}p{3cm}@{}}
Usiądź tu obok… \\ \\
\end{tabular}

\leftskip0cm
\noindent
\begin{tabular}{@{}p{8.50cm}p{3cm}@{}}
A kiedy przyjdziesz do mnie znów pojutrze \\
Popatrzysz smutno przez okno otwarte \\
Wsłuchasz się w miasta przedwieczorny szum \\
Powiem Ci – nie chodź tam, gdzie obcy tłum \\ \\
\end{tabular}

\leftskip1cm
\noindent
\begin{tabular}{@{}p{7.50cm}p{3cm}@{}} 
Siądziesz tu obok mnie w fotelu \\
Słonecznikowe pestki gryźć \\
A ja piosenkę Ci zanucę \\
Tę, której słuchać chciałabyś \\ \\
 
Może kałamarz strącisz z biurka \\
Lub z kalendarza zerwiesz dzień \\ 
Może już lubisz tę piosenkę \\
A może nawet lubisz mnie
\end{tabular}

\end{document}
