%w Z mega okna widać chmury na skalistych grzędach
%r Orawo, wiatrem malowany dach
\documentclass[a5paper]{article}
 \usepackage[english,bulgarian,russian,ukrainian,polish]{babel}
 \usepackage[utf8]{inputenc}
 %\usepackage{polski}
 \usepackage[T1]{fontenc}
 \usepackage[margin=1.5cm]{geometry}
 \usepackage{multicol}
 \setlength\columnsep{10pt}
 \begin{document}
 %\pagenumbering{gobble}


\noindent
\fontsize{12pt}{15pt}\selectfont
\textbf{Orawa} \\
\fontsize{8pt}{10pt}\selectfont
sł. i muz. Andrzej Wierzbicki \\ \\
\fontsize{10pt}{12pt}\selectfont
\leftskip0cm
\begin{tabular}{@{}p{9.00cm}p{3cm}@{}}
\noindent
wstęp & a C d E | x2 \\ \\
Z mego okna widać chmury na skalistych grzędach & a C d E \\
Przetrę szybę ciepłą dłonią, razem z nimi siędę & a C d E \\
I będą mi grały wiatry na organach turni & F C d E \\
Kiedy pójdę zbójnikować nad dachami równin & a C d E a \\ \\
 
Z mego okna widać potok - doliną, doliną \\
Dumnych smreków las szeroki, mgły w kosodrzewinach \\
I będą mi grały wiatry w zaklętych kolebach \\
Noc krzesanym się roztańczy - po niebach, po niebach \\ \\
\end{tabular}

\leftskip1cm
\noindent
\begin{tabular}{@{}p{8.00cm}p{3cm}@{}} 
Orawo, wiatrem malowany dach & F C d E \\
Ciupagami wysrebrzony na smrekowych pniach & C G H7 E \\
Orawo, wiatrem malowany dom & F C d E \\
Gdzie zbójnickie śpiewogrania po kolebach śpią & F C H7 a (w powtórze F C E7 a) \\ \\
\end{tabular}

\leftskip0cm
\noindent
\begin{tabular}{@{}p{8.00cm}p{3cm}@{}}
Z mego okna widać chmury na skalistych grzędach \\
Przetrę szybę ciepłą dłonią, razem z nimi siędę \\
I będą mi grały wiatry na organach turni \\
Moje życie tylko w górach, nad dachami równin \\ \\
\end{tabular}

\leftskip1cm
\noindent
\begin{tabular}{@{}p{8.00cm}p{3cm}@{}}
Orawo….
\end{tabular}

\end{document}
