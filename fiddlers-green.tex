%w Stary port się powoli układał do snu
%r Tylko wezmę mój sztormiak i sweter
\documentclass[a5paper]{article}
 \usepackage[english,bulgarian,russian,ukrainian,polish]{babel}
 \usepackage[utf8]{inputenc}
 %\usepackage{polski}
 \usepackage[T1]{fontenc}
 \usepackage[margin=1.5cm]{geometry}
 \usepackage{multicol}
 \setlength\columnsep{10pt}
 \begin{document}
 %\pagenumbering{gobble}


\noindent
\fontsize{12pt}{15pt}\selectfont
\textbf{Fiddler's green} \\
\fontsize{8pt}{10pt}\selectfont
sł. Jerzy Rogacki, muz. John Conolly \\ \\
\fontsize{10pt}{12pt}\selectfont
\leftskip0cm
\begin{tabular}{@{}p{9.00cm}p{3cm}@{}}
\noindent
Stary port się powoli układał do snu, & C F C a \\
Świeża bryza zmarszczyła morze gładkie jak stół, & C F C G  \\
Stary rybak na kei zaczął śpiewać swą pieśń: & F e d C \\
- Zabierzcie mnie chłopcy, mój czas kończy się. & a7 d7 F G \\ \\
\end{tabular}

\leftskip1cm
\noindent
\begin{tabular}{@{}p{8.00cm}p{3cm}@{}}
Tylko wezmę mój sztormiak i sweter, & C G C C7 \\
Ostatni raz spojrzę na pirs. & F C G \\
Pozdrów moich kolegów, powiedz, że dnia pewnego & F C e \\
Spotkamy się wszyscy tam w Fiddler's Green. & d G F C \\ \\
\end{tabular}

\leftskip0cm
\noindent
\begin{tabular}{@{}p{8.00cm}p{3cm}@{}}
O Fiddler's Green słyszałem nie raz,  \\
Jeśli piekło ominę, dopłynąć chcę tam, \\
Gdzie delfiny figlują w wodzie czystej jak łza, \\
A o mroźnej Grenlandii zapomina się tam. \\ \\

Kiedy już tam dopłynę, oddam cumy na ląd, \\
Różne bary są czynne cały dzień, całą noc, \\
Piwo nic nie kosztuje, dziewczęta jak sen, \\
A rum w buteleczkach rośnie na każdym z drzew. \\ \\

Aureola i harfa to nie to, o czym śnię, \\
O morza rozkołys i wiatr modlę się. \\
Stare pudło wyciągnę, zagram coś w cichą noc, \\
A wiatr w takielunku zaśpiewa swój song.
\end{tabular}

\end{document}
