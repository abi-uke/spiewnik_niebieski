%w Gdy byłem mały wciąż mi mówił tata
%r Ja chciałbym mieć czapkę z pomponem z boku
\documentclass[a5paper]{article}
 \usepackage[english,bulgarian,russian,ukrainian,polish]{babel}
 \usepackage[utf8]{inputenc}
 %\usepackage{polski}
 \usepackage[T1]{fontenc}
 \usepackage[margin=1.5cm]{geometry}
 \usepackage{multicol}
 \setlength\columnsep{10pt}
 \begin{document}
 %\pagenumbering{gobble}


\noindent
\fontsize{12pt}{15pt}\selectfont
\textbf{Pastuch} \\
\fontsize{8pt}{10pt}\selectfont
sł. i muz. Jaromir Nohavica, tłum. Antoni Muracki \\ \\
\fontsize{10pt}{12pt}\selectfont
\leftskip0cm
\begin{tabular}{@{}p{7.50cm}p{3cm}@{}}
\noindent
Gdy byłem mały wciąż mi mówił tata & C \\
że jeszcze zrobi ze mnie adwokata & C \\
więc paragrafy musiałem wbijać do głowy & F G C \\ \\

Taki adwokat grubą forsę kosi \\
siedzi w fotelu i dłubie palcem w nosie \\
a ja mu na to, że wolę wypasać krowy \\ \\
\end{tabular}

\leftskip1cm
\noindent
\begin{tabular}{@{}p{10.50cm}p{3cm}@{}}
Ja chciałbym mieć czapkę z pomponem z boku & C \\
jeść ulęgałki, pływać w potoku & C \\
i śpiewać przez cały dzień refrenik ten, & F G C \\
tak śpiewać: pam pam pam… & F G C \\ \\
\end{tabular}

\leftskip0cm
\noindent
\begin{tabular}{@{}p{10.50cm}p{3cm}@{}}
Stosy książek pod choinkę dawali mi \\
ale nadal nie umiałem odnaleźć w nich \\
prostej instrukcji – jak wypasa się krowy \\ \\

Pytałem starszych, lecz każdy się śmiał \\
i telefon do lekarza podać mi chciał \\
i pytał czy poza tym w domu wszyscy są zdrowi \\ \\
\end{tabular}

\leftskip1cm
\noindent
\begin{tabular}{@{}p{10.50cm}p{3cm}@{}}
Ja chciałbym mieć czapkę z pomponem z boku… \\ \\
\end{tabular}

\leftskip0cm
\noindent
\begin{tabular}{@{}p{10.50cm}p{3cm}@{}}
Dziś choć podrosłem i swoje już wiem \\
parę rzeczy mogę zmienić, a paru nie \\
to gdy mi smutno w mokrej kładę się trawie \\ \\

Z dłońmi za głową sobie leżę, a co! \\
gapię się w granatowe nieba tło, \\
gdzie wśród obłoków moje łaciate krowy się bawią
\end{tabular}

\end{document}
