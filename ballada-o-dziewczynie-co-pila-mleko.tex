%w Są małe stacje wielkich kolei
%r I nieraz chciałbym, żeby tu była
\documentclass[a5paper]{article}
 \usepackage[english,bulgarian,russian,ukrainian,polish]{babel}
 \usepackage[utf8]{inputenc}
 %\usepackage{polski}
 \usepackage[T1]{fontenc}
 \usepackage[margin=1.5cm]{geometry}
 \usepackage{multicol}
 \setlength\columnsep{10pt}
 \begin{document}
 %\pagenumbering{gobble}


\noindent
\fontsize{12pt}{15pt}\selectfont
\textbf{Ballada o dziewczynie co piła mleko} \\
\fontsize{8pt}{10pt}\selectfont
sł.: Agnieszka Osiecka, muz.: Katarzyna Sobolewska\\ \\
\fontsize{10pt}{12pt}\selectfont
\leftskip0cm
\begin{tabular}{@{}p{8.5cm}p{3cm}@{}}
\noindent
Są małe stacje wielkich kolei & D fis \\
Nieznane, jak obce imiona. & G D \\
Małe stacje wielkich kolei & G D \\
Jakiś napis i lampa zielona. & G A \\ \\

Na takiej stacji, dawno już temu & D fis \\
Z daleka jadąc daleko & G D \\
Widziałem dziewczynę & G \\
w niebieskim szaliku & D \\
Jak piła gorące mleko. & A D \\ \\

Teraz tamtędy nigdy nie jeżdżę & G D \\
I miasto moje daleko, & G D \\
A myślę czasem o tamtej dziewczynie & G D \\
Jak piła gorące mleko. & A D \\ \\
\end{tabular}

\leftskip1cm
\noindent
\begin{tabular}{@{}p{7.5cm}p{3cm}@{}}
I nieraz chciałbym, żeby tu była & G D \\
Może to miałoby sens. & G D \\
Jak ona śmiesznie to mleko piła & G D \\
Gapiąc się na mnie spod rzęs. & A D \\ \\
\end{tabular}

\leftskip0cm
\noindent
\begin{tabular}{@{}p{8.5cm}p{3cm}@{}}
Mam swoje sprawy, inne podróże\\
I nie tamtędy mi droga,\\
Lubię ulice wesołe i długie\\
I kolorowe światła na rogach.\\\\

Może ma chłopca tamta dziewczyna\\
Może wybrała się w świat,\\
A może po prostu jest taka głupia\\
Jak jej siedemnaście lat.\\\\

Zresztą to przecież nie ma znaczenia\\
Mieszkam naprawdę daleko,\\
A myślę czasem o tamtej dziewczynie\\
Jak piła gorące mleko.\\\\
\end{tabular}

\leftskip1cm
\noindent
\begin{tabular}{@{}p{8.5cm}p{3cm}@{}}
I nieraz…
\end{tabular}

\end{document}
