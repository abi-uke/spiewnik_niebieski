%w Co było nie wróci i szaty rozdzierać by próżno
\documentclass[a5paper]{article}
 \usepackage[english,bulgarian,russian,ukrainian,polish]{babel}
 \usepackage[utf8]{inputenc}
 %\usepackage{polski}
 \usepackage[T1]{fontenc}
 \usepackage[margin=1.5cm]{geometry}
 \usepackage{multicol}
 \setlength\columnsep{10pt}
 \begin{document}
 %\pagenumbering{gobble}


\noindent
\fontsize{12pt}{15pt}\selectfont
\textbf{Aleksander Siergiejewicz Puszkin} \\
\fontsize{8pt}{10pt}\selectfont
Bułat Okudżawa tłum. Witold Dąbrowski \\ \\
\fontsize{10pt}{12pt}\selectfont
\leftskip0cm
\begin{tabular}{@{}p{8.5cm}p{3cm}@{}}
\noindent
Co było - nie wróci i szaty rozdzierać by próżno. & a E7 \\
Cóż, każda epoka ma własny porządek i ład & C G7 C (A7) \\
A przecież mi żal, że tu w drzwiach & d \\
nie pojawi się Puszkin & G7 a \\
Tak chętnie bym dziś choć na kwadrans & a F7 E7 \\
na koniak z nim wpadł & a (A7) \\
A przecież mi żal, że tu w drzwiach & d \\
nie pojawi się Puszkin & G7 a \\
Tak chętnie bym dziś choć na kwadrans & a F7 E7 \\
na koniak z nim wpadł & E7 a E \\ \\
\end{tabular}

\noindent
\leftskip0cm
\begin{tabular}{@{}p{10cm}@{}}
\noindent
Dziś już nie musimy piechotą się wlec na spotkanie - \\
i tyle jest aut, i rakiety unoszą nas w dal\\
A przecież mi żal, że po Moskwie nie suną już sanie\\
I nie ma już sań, i nie będzie już nigdy, a żal\\ \\

Podziwiam i wielbię mój wiek, mego stwórcę i mistrza\\
Genialny mój wiek, piękny wiek pragnę cenić i czcić\\
A przecież mi żal, że jak dawniej śnią się nam bożyszcza\\
I jakoś tak jest, że gotowiśmy czołem im bić\\ \\

No cóż, nie na darmo zwycięstwem nasz szlak się uświetnił\\
I wszystko już jest, cicha przystań, non-iron i wikt\\
A przecież mi żal, że nad naszym zwycięstwem niejednym\\
Górują cokoły, na których nie stoi już nikt\\ \\

Co było, nie wróci, wychodzę wieczorem na spacer\\
I nagle spojrzałem na Arbat i - ach, co za gość!\\
Rżą konie u sań, Aleksander Siergiejewicz przechadza się\\
I głowę bym dał, że już jutro wydarzy się coś!
\end{tabular}

\end{document}
