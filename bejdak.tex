%w Skrajem nieba szedł Bejdak
%r Ktoś go poprosił by mu łąkę skosił
\documentclass[a5paper]{article}
 \usepackage[english,bulgarian,russian,ukrainian,polish]{babel}
 \usepackage[utf8]{inputenc}
 %\usepackage{polski}
 \usepackage[T1]{fontenc}
 \usepackage[margin=1.5cm]{geometry}
 \usepackage{multicol}
 \setlength\columnsep{10pt}
 \begin{document}
 %\pagenumbering{gobble}


\noindent
\fontsize{12pt}{15pt}\selectfont
\textbf{Bejdak} \\
\fontsize{8pt}{10pt}\selectfont
Stare Dobre Małżeństwo,  sł. A. Ziemianin, muz. K. Myszkowski \\ \\
\fontsize{10pt}{12pt}\selectfont
\leftskip0cm
\begin{tabular}{@{}p{8.5cm}p{3cm}@{}}
\noindent
Skrajem nieba szedł Bejdak & C \\
Organki z odpustu same mu grały & F C \\
Kałużę żabom łyżką zamieszał & d F \\
Nie zdając sobie z tego sprawy & G \\ \\

Podobno ktoś widział jak do żab się łasił \\
na kolanach żeby mu kumkały \\
Albo w pasiece do ula się spowiadał \\
A pszczoły miód na serce mu lały \\ \\
\end{tabular}

\leftskip1cm
\noindent
\begin{tabular}{@{}p{7.5cm}p{3cm}@{}}
Ktoś go poprosił by mu łąkę skosił & F \\
A on zioła głaskał, tulił się do trawy & C G \\
Jakoś nie umiał z ludźmi żyć & d F \\
Raczej kumplował się z ptakami & G \\
Kiedyś nad ranem z nimi odleciał & F \\
Na niebieskie ptasie polany & C G \\
Jakoś nie umiał z ludźmi żyć & d F G \\ \\
\end{tabular}

\leftskip0cm
\noindent
\begin{tabular}{@{}p{8.5cm}p{3cm}@{}}
Czasem wróblem wraca gdy Boga uprosi\\
Z lotu ptaka chwilę u nas gości\\
Boga słabość do niego jednaka\\
Bo jak nie kochać takiego Bejdaka\\\\
\end{tabular}

\leftskip1cm
\noindent
\begin{tabular}{@{}p{8.5cm}p{3cm}@{}}
Ktoś go poprosił by mu łąkę skosił…\\\\
\end{tabular}

\leftskip0cm
\noindent
\begin{tabular}{@{}p{8.5cm}p{3cm}@{}}
Skrajem nieba szedł Bejdak \\
Organki z odpustu same mu grały\\ 
Kałużę żabom łyżką zamieszał \\
Nie zdając sobie z tego sprawy
\end{tabular}

\end{document}
