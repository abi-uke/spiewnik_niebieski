%w Jak jest ziemia, co podpiera stopy
%r Jam jest nic i wiele
\documentclass[a5paper]{article}
 \usepackage[english,bulgarian,russian,ukrainian,polish]{babel}
 \usepackage[utf8]{inputenc}
 %\usepackage{polski}
 \usepackage[T1]{fontenc}
 \usepackage[margin=1.5cm]{geometry}
 \usepackage{multicol}
 \setlength\columnsep{10pt}
 \begin{document}
 %\pagenumbering{gobble}


\noindent
\fontsize{12pt}{15pt}\selectfont
\textbf{Jam / Wiosenny} \\
\fontsize{8pt}{10pt}\selectfont
Dom o Zielonych Progach / sł. i muz. A. Gniady \& Marcyś \\ \\
\fontsize{10pt}{12pt}\selectfont
\leftskip0cm
\begin{tabular}{@{}p{6.50cm}p{3cm}@{}}
\noindent
Jam jest ziemia, co podpiera stopy & C G \\
Jam jest wiatr, co rozwiewa włosy & a F G \\
Jam jest deszcz, co obmywa twarz & \\
Jam jest słońce, co wysusza skórę & \\ \\
\end{tabular}

\leftskip1cm
\noindent
\begin{tabular}{@{}p{6.50cm}@{}}
Jam jest nic i wiele \\
Jam jest ten, co płacze \\
i ten co się śmieje \\ \\
\end{tabular}

\leftskip0cm
\noindent
\begin{tabular}{@{}p{7.50cm}@{}}
On jest ciepłem, co rozgrzewa ciało \\
On jest chmurą, co osłania mnie \\
On jest mgłą, co łagodzi ranek \\
On jest ciepłem, które mieszka we mnie \\ \\
\end{tabular}

\leftskip1cm
\noindent
\begin{tabular}{@{}p{6.50cm}@{}}
On jest nic i wiele…
\end{tabular}

\end{document}
