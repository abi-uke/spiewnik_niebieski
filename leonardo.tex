%w Ja nie wesoła, ale z kokardą
%r Dość jest wszystkiego, dojść można wszędzie
\documentclass[a5paper]{article}
 \usepackage[english,bulgarian,russian,ukrainian,polish]{babel}
 \usepackage[utf8]{inputenc}
 %\usepackage{polski}
 \usepackage[T1]{fontenc}
 \usepackage[margin=1.5cm]{geometry}
 \usepackage{multicol}
 \setlength\columnsep{10pt}
 \begin{document}
 %\pagenumbering{gobble}


\noindent
\fontsize{12pt}{15pt}\selectfont
\textbf{Leonardo} \\
\fontsize{8pt}{10pt}\selectfont
Jacek Kleyff \\ Maryla Rodowicz\\
\fontsize{10pt}{12pt}\selectfont
\leftskip0cm
\begin{tabular}{@{}p{7.50cm}p{3cm}@{}}
\noindent
Ja nie wesoła, ale z kokardą & C G \\
Lecę do słońca, hej! Leonardo & a F G \\
A ja się kręcę, bo stać nie warto \\
Naprzód planeto, hej! Leonardo \\ \\
\end{tabular}

\leftskip1cm
\noindent
\begin{tabular}{@{}p{4.00cm}p{3cm}@{}}
Dość jest wszystkiego \\
Dojść można wszędzie  \\ \\
\end{tabular}

\leftskip0cm
\noindent
\begin{tabular}{@{}p{6.50cm}p{3cm}@{}}
Diabeł mnie szarpie, trzyma za uszy,  \\
Dokąd wariatko, chcesz z nim wyruszyć \\
A ja gotowa, ja z halabardą \\
Hej! droga wolna, hej! Leonardo \\ \\

Panie w koronie, panie z liczydłem\\
Nie chcę być mrówką, ja chcę być skrzydłem\\
A moja głowa, droga i muzyka\\
Do brązowego życia umyka \\ \\

Wyszłam z bylekąd, ale co z tego \\
Zmierzam daleko, hej, hej kolego\\
Odłóżmy sprawy, kochany synku\\
Na jakieś dziewięć miejsc po przecinku\\ \\

Może to bujda, może to obłuda\\
Ale pasuje jej to jak ulał
\end{tabular}

\end{document}
