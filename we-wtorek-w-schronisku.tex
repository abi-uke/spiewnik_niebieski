%w Złotym kobiercem wymoszczone góry
\documentclass[a5paper]{article}
 \usepackage[english,bulgarian,russian,ukrainian,polish]{babel}
 \usepackage[utf8]{inputenc}
 %\usepackage{polski}
 \usepackage[T1]{fontenc}
 \usepackage[margin=1.5cm]{geometry}
 \usepackage{multicol}
 \setlength\columnsep{10pt}
 \begin{document}
 %\pagenumbering{gobble}


\noindent
\fontsize{12pt}{15pt}\selectfont
\textbf{We wtorek w schronisku} \\
\fontsize{8pt}{10pt}\selectfont
sł. W. Buchcic, muz. R. Pomorski \\ \\
\fontsize{10pt}{12pt}\selectfont
\leftskip0cm
\begin{tabular}{@{}p{8.50cm}p{3cm}@{}}
\noindent
Złotym kobiercem wymoszczone góry & C F C \\
Jesień w doliny przyszła dziś nad ranem & e F d G\\ 
Buki czerwienią zabarwiły chmury & C F E7 a \\
Z latem się złotym właśnie pożegnałem & F G C G \\ \\
\end{tabular}

\leftskip1cm
\noindent
\begin{tabular}{@{}p{7.50cm}p{3cm}@{}}
We wtorek w schronisku po sezonie & C F G C \\
W doliny wczoraj zszedł ostatni gość & a D7 G \\
Za oknem plucha, kubek parzy w dłonie & C F E7 a \\
I tej herbaty, i tych gór mam dość & F G C \\ \\
\end{tabular}

\leftskip0cm
\noindent
\begin{tabular}{@{}p{8.50cm}p{3cm}@{}}
Szaruga niebo powoli zasnuwa \\
Wiatr już gałęzie pootrząsał z liści \\
Pod wiatr, pod górę znowu sam zasuwam \\
Może w schronisku spotkam kogoś z bliskich \\ \\
\end{tabular}

\leftskip1cm
\noindent
\begin{tabular}{@{}p{7.50cm}@{}}
We wtorek w schronisku… \\ \\
\end{tabular}

\leftskip0cm
\noindent
\begin{tabular}{@{}p{8.50cm}@{}}
Ludzie tak wiele spraw muszą załatwić \\
A czas sobie płynie wolno panta rei \\
Do ciebie tylko już nie umiem trafić \\
Kochać to więcej siebie dać, czy mniej \\ \\
\end{tabular}

\leftskip1cm
\noindent
\begin{tabular}{@{}p{7.50cm}@{}}
We wtorek w schronisku…
\end{tabular}

\end{document}
