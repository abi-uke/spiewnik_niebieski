%w W szale skrzypki trzy strzaskał mistrz
\documentclass[a5paper]{article}
 \usepackage[english,bulgarian,russian,ukrainian,polish]{babel}
 \usepackage[utf8]{inputenc}
 %\usepackage{polski}
 \usepackage[T1]{fontenc}
 \usepackage[margin=1.5cm]{geometry}
 \usepackage{multicol}
 \setlength\columnsep{10pt}
 \begin{document}
 %\pagenumbering{gobble}


\noindent
\fontsize{12pt}{15pt}\selectfont
\textbf{Mistrz skrzypiec z Przasnysza} \\
\fontsize{8pt}{10pt}\selectfont
autor nieznany \\ \\
\fontsize{10pt}{12pt}\selectfont
\leftskip0cm
\begin{tabular}{@{}p{8.50cm}p{3cm}@{}}
\noindent
W szale skrzypki trzy strzaskał mistrz & a \\
Trzasnął drzwiami też - zgrzyt mu zbrzydł & \\
Przeciez sprzęt przestał brzmieć & F G \\
Przez przypadek starzec sczezł & C a \\
Zebrząc żre zżuty strzęp & d a \\
Przy użyciu sztucznych szczęk & E a \\ \\
\end{tabular}

\leftskip0cm
\noindent
\begin{tabular}{@{}p{8.50cm}p{3cm}@{}}
Aż raz rzecze mistrz: Przasnych znasz & a \\
Żółty żupan włóż, przetrzyj twarz & \\
Sprzedaj trzos wstrzymaj łzy & F G \\
Możesz brzytwą zarost strzyc & C a \\
Przepasz brzuch, przeżyj wstrząs & d a \\
Brzaskiem wrzesień srebrny wrzos. & E a
\end{tabular}

\end{document}
