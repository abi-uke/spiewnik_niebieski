%w Chleba takiego jak ten od Cześka
%r Chleb się, chlebie
\documentclass[a5paper]{article}
 \usepackage[english,bulgarian,russian,ukrainian,polish]{babel}
 \usepackage[utf8]{inputenc}
 %\usepackage{polski}
 \usepackage[T1]{fontenc}
 \usepackage[margin=1.5cm]{geometry}
 \usepackage{multicol}
 \setlength\columnsep{10pt}
 \begin{document}
 %\pagenumbering{gobble}


\noindent
\fontsize{12pt}{15pt}\selectfont
\textbf{Ballada o Cześku piekarzu} \\
\fontsize{8pt}{10pt}\selectfont
Wolna Grupa Bukowina, sł. i muz. W. Bellon \\ \\
\fontsize{10pt}{12pt}\selectfont
\leftskip0cm
\begin{tabular}{@{}p{8.5cm}p{3cm}@{}}
\noindent
Chleba takiego jak ten od Cześka & D A \\
Nie kupisz nigdzie, nawet w Warszawie, & e G Fis h \\
Bo Czesiek piekarz nie piekł, lecz tworzył & G A \\
Bochny jak z mąki słonecznej kołacze. & D A \\ \\

Kłaniali mu się ludzie, gdy wyjrzał & D A \\
Przez okno w kitlu łyknąć powietrza, & e G Fis h \\
A kromkę masłem smarując każdy & G A \\
Mówił – nad chleby ten chleb od Cześka. & D A D\\ \\
\end{tabular}

\leftskip1cm
\noindent
\begin{tabular}{@{}p{7.5cm}p{3cm}@{}}
Chleb się, chlebie, chleb się, chlebie & C G e a \\
Bo nad chleb być może co? & C h e \\
Chleb się, chlebie, chleb się, chlebie & C G e a \\
Niech ci nigdy nie zabraknie & C \\
Drożdży, wody, rąk i ziarna & D A e h \\
Mruczał Czesiek tak noc w noc. & G D A G D \\ \\
\end{tabular}

\leftskip0cm
\noindent
\begin{tabular}{@{}p{8.5cm}p{3cm}@{}}
A o porankach chlebem pachnących, \\
Gdy pora idzie spać na piekarzy \\
Zaczerwienione przymykał oczy \\
Czesiek i siadał z dłutem przy stole. \\ \\
 
Ciągle te same włosy i trochę \\
Za duży nos w drewnie cierpliwym. \\
Pieściły ręce dziesiątki razy \\
W poranki świeżym chlebem pachnące. \\ \\
\end{tabular}

\leftskip1cm
\noindent
\begin{tabular}{@{}p{8.5cm}p{3cm}@{}} 
Chleb się, chlebie… \\ \\
\end{tabular}

\leftskip0cm
\noindent
\begin{tabular}{@{}p{8.5cm}p{3cm}@{}}
Nikt takich słów jak miasto miastem \\
Nie znał – i źle się dzieje, mówili. \\
Na obraz czerniał Czesiek razowca \\
Kruszał podobnie bułce zleżałej. \\ \\
 
Gdy go znaleźli na pasku z wojska \\
Dłuto jak w bochen wbite miał w garści \\
I nie wie nikt, co Cześka wzięło \\
Lecz śpiewa każdy, jak miasto miastem.
\end{tabular}

\end{document}
