\documentclass[a5paper]{article}
 \usepackage[english,bulgarian,russian,ukrainian,polish]{babel}
 \usepackage[utf8]{inputenc}
 %\usepackage{polski}
 \usepackage[T1]{fontenc}
 \usepackage[margin=1.5cm]{geometry}
 \usepackage{multicol}
 \setlength\columnsep{10pt}
 \begin{document}
 %\pagenumbering{gobble}


\noindent
\fontsize{12pt}{15pt}\selectfont
\textbf{Czom ty ne pryjszoł} \\
\fontsize{8pt}{10pt}\selectfont
Dom o zielonych progach \\ \\
\fontsize{10pt}{12pt}\selectfont
\leftskip0cm
\begin{tabular}{@{}p{6.50cm}p{3cm}@{}}
\noindent
Czom ty ne pryjszoł & a \\
Jak misjac zijszoł & a \\
Ja tebe czekała	& C G C A7 \\
Czy konia ne mał & d\\
Czy steżky ne znał & C e a \\
Maty ne puskała	& C E E7 a A7 \\
Czy konia ne mał & d \\
Czy steżky ne znał & C e a \\
Maty ne puskała	& C E E7 a \\ \\
\end{tabular}

\leftskip0cm
\noindent
\begin{tabular}{@{}p{8.50cm}@{}}
I konia ja mał \\
I steżky ja znał \\
I maty puskała \\
Najmensza sestra \\
Bodaj ne zrosła \\
Sidelce schowała \\ \\
\end{tabular}

\leftskip0cm
\noindent
\begin{tabular}{@{}p{8.50cm}@{}}
A starsza sestra \\
Sidelce znajszła \\
Konia osidłała \\
Pojid brateńku \\
Do diłczynońky \\
Szczo by ne czekała \\ \\
\end{tabular}

\leftskip0cm
\noindent
\begin{tabular}{@{}p{8.50cm}@{}}
Tecze riczeńka \\
Newełyczeńka \\
Schoczu, pereskoczu \\
Widdajte mene \\
Moja matinko \\
Za koho ja choczu \\ \\
\end{tabular}

\leftskip0cm
\noindent
\begin{tabular}{@{}p{8.50cm}@{}}
Dum, dum, dum, dum, dum \\
Dum, dum, dum, dum, dum \\
Dum, dum, dum, dum, dum, dum, dum \\
Widdajte mene \\
Moja matinko \\
Za koho ja choczu \\
\end{tabular}
\end{document}
