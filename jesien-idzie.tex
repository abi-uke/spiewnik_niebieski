%w Raz staruszek spacerując w lesie
\documentclass[a5paper]{article}
 \usepackage[english,bulgarian,russian,ukrainian,polish]{babel}
 \usepackage[utf8]{inputenc}
 %\usepackage{polski}
 \usepackage[T1]{fontenc}
 \usepackage[margin=1.5cm]{geometry}
 \usepackage{multicol}
 \setlength\columnsep{10pt}
 \begin{document}
 %\pagenumbering{gobble}


\noindent
\fontsize{12pt}{15pt}\selectfont
\textbf{Jesień idzie} \\
\fontsize{8pt}{10pt}\selectfont
sł. Andrzej Waligórski, muz. Olek Grotowski \\ \\
\fontsize{10pt}{12pt}\selectfont
\leftskip0cm
\begin{tabular}{@{}p{7.50cm}p{3cm}@{}}
\noindent
Raz staruszek, spacerując w lesie, & e A7 e A7 \\
Ujrzał listek przywiędły i blady & e A7 H7 \\
I pomyślał: - Znowu idzie jesień, & e A7 e\\
Jesień idzie, nie ma na to rady! & C H7 e A7 \\ \\
\end{tabular} 

\leftskip0cm
\noindent
\begin{tabular}{@{}p{7.50cm}p{3cm}@{}}
I podreptał do chaty po dróżce,	& C D G e \\
I oznajmił, stanąwszy przed chatą & C D G e \\
Swojej żonie, tak samo staruszce & C D G e \\
Jesień idzie, nie ma rady na to! & C H7 e A7 e H7 \\ \\
\end{tabular}

\leftskip0cm
\noindent
\begin{tabular}{@{}p{8.50cm}@{}}
A staruszka zmartwiła się szczerze, \\
Zamachała rękami obiema: \\
- Musisz zacząć chodzić w pulowerze. \\
Jesień idzie, rady na to nie ma! \\ \\
\end{tabular}

\leftskip0cm
\noindent
\begin{tabular}{@{}p{8.50cm}@{}}
Może zrobić się chłodno już jutro \\
Lub pojutrze, a może za tydzień?  \\
Trzeba będzie wyjąć z kufra futro, \\
Nie ma rady, jesień, jesień idzie! \\ \\
\end{tabular}

\leftskip0cm
\noindent
\begin{tabular}{@{}p{8.50cm}@{}}
A był sierpień. Pogoda prześliczna. \\
Wszystko w złocie trwało i w zieleni. \\
Prócz staruszków nikt chyba nie myślał \\
O mającej nastąpić jesieni. \\ \\
\end{tabular}

\leftskip0cm
\noindent
\begin{tabular}{@{}p{8.50cm}@{}}
Ale cóż, oni żyli najdłużej, \\
Mieli swoje staruszkowe zasady \\
I wiedzieli, że wcześniej czy później - \\
Jesień przyjdzie. Nie ma na to rady. 
\end{tabular}
\end{document}
