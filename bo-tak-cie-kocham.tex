%w Gdy już będziemy wraz na grzebietach dzikich koni
%r Tak Cię kocham jak nikogo na Ziemi
\documentclass[a5paper]{article}
 \usepackage[english,bulgarian,russian,ukrainian,polish]{babel}
 \usepackage[utf8]{inputenc}
 %\usepackage{polski}
 \usepackage[T1]{fontenc}
 \usepackage[margin=1.5cm]{geometry}
 \usepackage{multicol}
 \setlength\columnsep{10pt}
 \begin{document}
 %\pagenumbering{gobble}


\noindent
\fontsize{12pt}{15pt}\selectfont
\textbf{Bo tak Cię kocham} \\
\fontsize{8pt}{10pt}\selectfont
Cisza Jak Ta / sł. M.Łangowski, muz. M.Łangowski, T.Fojgt \\ \\
\fontsize{10pt}{12pt}\selectfont
\leftskip0cm
\begin{tabular}{@{}p{9cm}p{2cm}@{}}
\noindent
Gdy już będziemy wraz na grzbietach dzikich koni & a G C F \\
ku połoninie gnać, spłoszone ptaki gonić & e F a G \\
I myśli, płoche słowa ubierać w zieleń drzew & F G C a \\
Gdy wtedy spytasz o czym myślę – odpowiedź wplotę & \\
~~~~~~~~~~~~~~~~~~~~~~~~~~~~~~~~~~~~~~~~~~~~~~~~~~~~~~~~w wiatru śpiew & F G F C \\ \\
\end{tabular}

\leftskip1cm
\noindent
\begin{tabular}{@{}p{8cm}p{2cm}@{}}
Tak Cię kocham & F G \\
jak nikogo na Ziemi & C a \\
w obłoki frunie każda myśl & F G \\
każda myśl zielona & d7 G \\ \\

Tak Cię kocham & F G \\
i to już się nie zmieni & C a \\
niech te słowa w Twoje włosy & F G \\
wplecie wiatr & F C \\ \\
\end{tabular}

\leftskip0cm
\noindent
\begin{tabular}{@{}p{10.5cm}@{}}
Gdy kolorowe będą światy, i złote liście pokryją szlak \\
tak pięknie się w źrenicach mienią nie mogąc się nadziwić jak \\
Kwietniowa zieleń Twoich oczu z wrześniową tęczą łączy się \\
ja chciałbym tonąć w tej zieleni i po tej tęczy z Tobą biec \\ \\

Gdy ponad dachy szarych miast, chmur ciężkich zimna moc napłynie \\
by zetrzeć naszych wspomnień czar nasz pierwszy świt na połoninie \\
A wokół pełen prozy świat zabrania snom się wciąż zielenić \\
Lecz co połączył górski wiatr – tego już nikt nie może zmienić \\ \\
\end{tabular}

\leftskip1cm
\noindent
\begin{tabular}{@{}p{8.5cm}@{}}
Bo tak Cię kocham… \\ \\
Tak Cię kocham…
\end{tabular}

\end{document}
