\documentclass[a5paper]{article}
 \usepackage[english,bulgarian,russian,ukrainian,polish]{babel}
 \usepackage[utf8]{inputenc}
 %\usepackage{polski}
 \usepackage[T1]{fontenc}
 \usepackage[margin=1.5cm]{geometry}
 \usepackage{multicol}
 \setlength\columnsep{10pt}
 \begin{document}
 %\pagenumbering{gobble}


\noindent
\fontsize{12pt}{15pt}\selectfont
\textbf{Góry i ludzie} \\
\fontsize{8pt}{10pt}\selectfont
Dom o Zielonych Progach / sł. J. Harasymowicz, muz. tradycyjna \\ \\
\fontsize{10pt}{12pt}\selectfont
\leftskip0cm
\begin{tabular}{@{}p{7.50cm}p{3cm}@{}}
\noindent
Góry i ludzie z nieba schodzą, & E fis \\
Trochanowscy prowadzą basy. & A E \\
Przez wieś wiodą jak niedźwiedzia, & \\
Jak jastrząb skrzypeczki płaczą. & \\ \\
\end{tabular}

\leftskip0cm
\noindent
\begin{tabular}{@{}p{7.50cm}@{}}
To Sławko gra skrzypce trzyma, \\
smykiem zahacza o szczyty dalsze. \\
A bas Piotra jak wicher wydyma \\
banie na cerkwi na cerkwi w Bielance. \\ \\

Góry i ludzie z nieba schodzą, \\
Na drodze życzliwa życzliwa ciemność. \\
Wreszcie po latach tych przy stole \\
zsiadło się ze mną morze Łemków. \\ \\

Jedna nad nami Łemkowyna \\
i jeden Święty Jerzy czuwa. \\ 
Jezus na tronie wypoczywa, \\
czesany w koki jak Samuraj.
\end{tabular}

\end{document}
