%w Warszawiacy śpiewają o Wiśle
\documentclass[a5paper]{article}
 \usepackage[english,bulgarian,russian,ukrainian,polish]{babel}
 \usepackage[utf8]{inputenc}
 %\usepackage{polski}
 \usepackage[T1]{fontenc}
 \usepackage[margin=1.5cm]{geometry}
 \usepackage{multicol}
 \setlength\columnsep{10pt}
 \begin{document}
 %\pagenumbering{gobble}


\noindent
\fontsize{12pt}{15pt}\selectfont
\textbf{Jeśli chcecie gdzieś przenosić to w Bieszczady (Stolica)} \\
\fontsize{8pt}{10pt}\selectfont
sł. Artur Andrus, muz. Olek Grotowski \\ \\
\fontsize{10pt}{12pt}\selectfont
\leftskip0cm
\begin{tabular}{@{}p{9.5cm}p{3cm}@{}}
\noindent
Warszawiacy śpiewają o Wiśle & D \\
Ci z Olsztyna śpiewają o Łynie & G \\
Ci z Katowic o ciężkim przemyśle & e \\
My śpiewamy o tamie w Solinie & D G A \\
Tutaj wszystko ładniejsze i lepsze & D \\
W porównaniu na przykład z Krakowem & G \\
Czysta woda i czyste powietrze & e \\
Choć pod Jasłem są pola naftowe & D G A \\ \\
\end{tabular}

\leftskip0cm
\noindent
\begin{tabular}{@{}p{9.5cm}p{3cm}@{}}
Wielkie nieba co za wrzawa & G D G D \\
chcą przenosić ale gdzie & G A D \\
Olsztym Kraków czy Warszawa & G D \\
a ja jednak myślę, że & G e A \\ \\
\end{tabular}

\leftskip1cm
\noindent
\begin{tabular}{@{}p{8.5cm}p{3cm}@{}}
Jeśli chcecie gdzieś przenosić to w Bieszczady & G A D \\
Bo to idealne miejsce dla stolicy & G A D \\
Komuniści z ludowcami mogą wchodzić tu w układy & G A G e \\
W razie czego mają blisko do granicy & D A D \\ \\
\end{tabular}

\leftskip1cm
\noindent
\begin{tabular}{@{}p{8.5cm}p{3cm}@{}}
Tu najlepszy najżyczliwszy żyje naród & G A D \\
I nieprawda że tu bieda głód i nędza & G A D \\
Na mieszkańca nam przypada samej ziemi 6 hektarów & G A G A \\
Tona drewna, niedźwiedź i 2/3 księdza. & D A D \\ \\
\end{tabular}

\leftskip0cm
\noindent
\begin{tabular}{@{}p{9.5cm}p{3cm}@{}}
Spoglądają krakowcy poeci & D \\
Na poetów warszawskich z wysoka & G \\
Lecz przypomnę, panowie że przecież & e \\
Już przed wami był Grzegorz z Sanoka & D G A \\
Po tych ścieżkach pan szwejk spacerował & D \\
Był Jagiełło z małżonką i świtą & G \\
Przebywała tu Bona królowa & e \\
Oraz Gierek z Josipem Rostitą & D G A \\ \\
\end{tabular}

\leftskip0cm
\noindent
\begin{tabular}{@{}p{9.5cm}p{3cm}@{}}
Wiatr się tłucze po Tarnicy & G D G D \\
otulonej płaszczem gwiazd & G A D \\
Ale wrócimy do stolicy & e D \\
i powtórzmy jeszcze raz & G e A \\
\end{tabular}

\leftskip1cm
\noindent
\begin{tabular}{@{}p{8.5cm}p{3cm}@{}}
Jeśli przeneiść to w Bieszczady bo panowie & G A D \\
Każdy z was odnzjadzie tutaj coś swojskiego & G A D \\
Jak przed laty w gęstym lesie stoją dacze w Warłamowie & G A G e \\
A w Jabłonkach stoi pomnik Świerczewskiego & D A D \\ \\
\end{tabular}

\leftskip1cm
\noindent
\begin{tabular}{@{}p{8.5cm}p{3cm}@{}}
Tu najlepszy najżyczliwszy żyje naród & G A D \\
I nieprawda że tu bieda głód i nędza & G A D \\
Na mieszkańca nam przypada samej ziemi 6 hektarów & G A G A \\
Tona drewna, niedźwiedź i 2/3 księdza. & D A D \\ \\
\end{tabular}

\leftskip0cm
\noindent
\begin{tabular}{@{}p{9.5cm}p{3cm}@{}}
Więc przenieście tę stolicę pod Tarnicę & G A D \\
Gdzie nocami diabeł z lichem w karty rżnie & G A D \\
Prosi o to Małgorzata dziecko Łodzi robotniczej & G A G e \\
I stoczniowiec z Gdańska Aleksander G. & D A D
\end{tabular}

\end{document}
